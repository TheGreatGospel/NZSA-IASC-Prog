\documentclass[12pt]{article}
% \documentstyle{iascars2017}

% \usepackage{iascars2017}

\pagestyle{myheadings} 
\pagenumbering{arabic}
\topmargin 0pt \headheight 23pt \headsep 24.66pt
%\topmargin 11pt \headheight 12pt \headsep 13.66pt
\parindent = 3mm 


\begin{document}


\begin{flushleft}


{\LARGE\bf The Uncomfortable Entrepreneurs: Bad Working Conditions and Entrepreneurial Commitment}


\vspace{1.0cm}

Catherine Laffineur$^1$ 

\begin{description}

\item $^1 \;$ Department of Economics, University Cote d'Azur, National Center of Scientific Research, GREDEG-CNRS, France


\end{description}

\end{flushleft}

%  ***** ADD ENOUGH VERTICAL SPACE HERE TO ENSURE THAT THE *****
%  ***** ABSTRACT (OR MAIN TEXT) STARTS 5 CM BELOW THE TOP *****

\vspace{0.75cm}

\noindent {\bf Abstract}. In contrast to previous model dividing necessity entrepreneurs as individuals facing push factors due to lack of employment, we consider the possibility of push factors faced by employed individuals (Folta et al. (2010)). The theoretical model yields distinctive predictions relating occupation characteristics and the probability of entry into entrepreneurship. Using PSED and ONET data, we investigate how the characteristics of individuals? primary occupations affect nascent entrepreneurs? effort put into venture creation. The empirical evidences show that necessity entrepreneurs are not only confined to unemployed individuals. We find compelling evidence that individuals facing arduous working conditions (e.g. stressful environment and physical tiredness) have a higher likelihood of entering and succeeding in self-employment than others. Contrariwise, individuals who experience high degree of self-realization, independence and responsibility in the workplace are less committed to their business than individuals exposed to arduous working conditions. These findings have strong implication for how we interpret and analyze necessity entrepreneurs and provide novel insights into the role of occupational experience in the process of venture emergence. 

\vskip 2mm

\noindent {\bf Keywords}.
Keywords: Entrepreneurship, Motivation, Occupational characteristics, Employment choice.


%\section{ First-level heading}
%The C98 head 1 style leaves a half-line spacing below a
%first-level heading. There should be one blank line above
%a first-level heading.
%        
%\subsection { Second-level heading}
%There should also be one blank line above a second- or
%third-level heading (but no extra space below them).
%
%Do not intent the first paragraph following a heading.
%Second and subsequent paragraphs are indented by one Tab
%character (= 3 mm). If footnotes are used, they should be
%placed at the foot of the page\footnote{ Footnotes are separated
%from the text by a blank line and a printed line of length 3.5 cm.
%They should be printed in 9-point Times Roman in single line spacing.}.
%        
%\subsubsection { Third-level heading}
%Please specify references using the conventions
%illustrated below. Each should begin on a new line, and
%second and subsequent lines should be on the same page
%indented by 3 mm.

\subsection*{References}

\begin{description}

\item Folta, T. B., Delmar, F., & Wennberg, K. 2010. Hybrid entrepreneurship. \textit{Management Science}, 56(2), 253-269.


\end{description}

\end{document}





