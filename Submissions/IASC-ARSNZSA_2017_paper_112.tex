\documentclass[12pt]{article}
% \documentstyle{iascars2017}

% \usepackage{iascars2017}

\pagestyle{myheadings} 
\pagenumbering{arabic}
\topmargin 0pt \headheight 23pt \headsep 24.66pt
%\topmargin 11pt \headheight 12pt \headsep 13.66pt
\parindent = 3mm 


\begin{document}


\begin{flushleft}


{\LARGE\bf Analysis of Official Microdata Using Secure Statistical Computation System
}


\vspace{1.0cm}

Kiyomi SHIRAKAWA$^1$,
Koji CHIDA$^2$,
Satoshi TAKAHASHI$^2$,
Satoshi TANAKA$^2$,
Ryo KIKUCHI$^2$ and
Dai IKARASHI$^2$

\begin{description}

\item $^1 \;$ National Statistics Center, Japan / Hitotsubashi University

\item $^2 \;$ NTT Secure Platform Laboratories

\end{description}

\end{flushleft}

%  ***** ADD ENOUGH VERTICAL SPACE HERE TO ENSURE THAT THE *****
%  ***** ABSTRACT (OR MAIN TEXT) STARTS 5 CM BELOW THE TOP *****

\vspace{0.75cm}

\noindent {\bf Abstract}.
We introduce some important functions on a secure computation system and empirically evaluate them using the statistical computing software R. The secure computation is a cryptographic technology that enables us to operate data while keeping the data encrypted. Due to the remarkable aspect, we can construct a secure on-line analytical system to protect against unauthorized access, computer virus and internal fraud. Moreover, the function of secure computation has a benefit for privacy. 

So far, we developed a secure computation system that runs R as a front-end application. In this research, we focus on the analysis of official microdata using our secure computation system. By employing the R script language to secure computation, we can potentially make new functions for the analysis of official microdata on our secure computation system. We show some examples of functions on the system using the R script language. A demonstration experiment to verify the practicality and scalability of the system in the field of official statistics is also in our scope.


\vskip 2mm

\noindent {\bf Keywords}.
Secure Computation, Security, Privacy, Big Data, Official Statistics, R

%\section{ First-level heading}
%The C98 head 1 style leaves a half-line spacing below a
%first-level heading. There should be one blank line above
%a first-level heading.
%        
%\subsection { Second-level heading}
%There should also be one blank line above a second- or
%third-level heading (but no extra space below them).
%
%Do not intent the first paragraph following a heading.
%Second and subsequent paragraphs are indented by one Tab
%character (= 3 mm). If footnotes are used, they should be
%placed at the foot of the page\footnote{ Footnotes are separated
%from the text by a blank line and a printed line of length 3.5 cm.
%They should be printed in 9-point Times Roman in single line spacing.}.
%        
%\subsubsection { Third-level heading}
%Please specify references using the conventions
%illustrated below. Each should begin on a new line, and
%second and subsequent lines should be on the same page
%indented by 3 mm.

%\subsection*{References}

%\begin{description}

%\item
%Barnett, J.A., Payne, R.W. and Yarrow, D. (1990).
%\textit{Yeasts: Characteristics and identification: Second Edition.}
%Cambridge: Cambridge University Press.

%\item
%(ed.) Barnett, V., Payne, R. and Steiner, R. (1995).
%\textit{Agricultural Sustainability: Economic, Environmental and
%Statistical Considerations}. Chichester: Wiley.

%\item
%Payne, R.W. (1997).
%\textit{Algorithm AS314 Inversion of matrices Statistics},
%\textbf{46}, 295--298.

%\item
%Payne, R.W. and Welham, S.J. (1990).
%A comparison of algorithms for combination of information in generally
%balanced designs.
%In: \textit{COMPSTAT90 Proceedings in Computational Statistics}, 297--302.
%Heidelberg: Physica-Verlag.

%\end{description}

\end{document}





