\documentclass[12pt]{article}
% \documentstyle{iascars2017}

% \usepackage{iascars2017}

\pagestyle{myheadings} 
\pagenumbering{arabic}
\topmargin 0pt \headheight 23pt \headsep 24.66pt
%\topmargin 11pt \headheight 12pt \headsep 13.66pt
\parindent = 3mm 


\begin{document}


\begin{flushleft}


{\LARGE\bf Meta-analysis with symbolic data analysis and its application for clinical data}


\vspace{1.0cm}

Ryo Takagi$^1$, Hiroyuki Minami$^2$, and Masahiro Mizuta$^2$

\begin{description}

\item $^1 \;$ Graduate School of Information Science and Technology, Hokkaido University

\item $^2 \;$ Laboratory of Advanced Data Science, Information Initiative Center, Hokkaido University

\end{description}

\end{flushleft}

%  ***** ADD ENOUGH VERTICAL SPACE HERE TO ENSURE THAT THE *****
%  ***** ABSTRACT (OR MAIN TEXT) STARTS 5 CM BELOW THE TOP *****

\vspace{0.75cm}

\noindent {\bf Abstract}. We discuss a method of meta-analysis based on symbolic data analysis (SDA). Meta-analysis, mainly used in social and medical science, is a statistical method of combining scientific studies to obtain quantitative results and provides a high level of evidence. Differences between the studies are caused by heterogeneity between the studies. It is useful to detect relationship among scientific studies. A target of analysis on SDA is {\it concept}, a set of individuals. We apply SDA to meta-analysis. In other words, we regard scientific studies as concepts. For example, symbolic clustering or symbolic MDS are useful to preprocess the scientific studies in meta-analysis. In this study, we propose a new approach based on SDA for meta-analysis and show the results of the proposed approach using clinical datasets.

\vskip 2mm

\noindent {\bf Keywords}.
symbolic clustering, symbolic MDS, concept in SDA


%\section{ First-level heading}
%The C98 head 1 style leaves a half-line spacing below a
%first-level heading. There should be one blank line above
%a first-level heading.
%        
%\subsection { Second-level heading}
%There should also be one blank line above a second- or
%third-level heading (but no extra space below them).
%
%Do not intent the first paragraph following a heading.
%Second and subsequent paragraphs are indented by one Tab
%character (= 3 mm). If footnotes are used, they should be
%placed at the foot of the page\footnote{ Footnotes are separated
%from the text by a blank line and a printed line of length 3.5 cm.
%They should be printed in 9-point Times Roman in single line spacing.}.
%        
%\subsubsection { Third-level heading}
%Please specify references using the conventions
%illustrated below. Each should begin on a new line, and
%second and subsequent lines should be on the same page
%indented by 3 mm.

\subsection*{References}

\begin{description}

\item
Edwin Diday and Monique Noirhomme-Fraiture. (2008).
\textit{Symbolic data analysis and the SODAS software.}
John Wiley \& Sons, Ltd.

\item
David Edward Matthews and Vernon Todd Farewell. (2015).
\textit{Using and understanding medical statistics} (5th, revised and extended edition).
Karger Publishers.

\end{description}

\end{document}





