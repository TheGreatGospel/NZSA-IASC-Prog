\documentclass[12pt]{article}
% \documentstyle{iascars2017}

% \usepackage{iascars2017}

\pagestyle{myheadings} 
\pagenumbering{arabic}
\topmargin 0pt \headheight 23pt \headsep 24.66pt
%\topmargin 11pt \headheight 12pt \headsep 13.66pt
\parindent = 3mm 


\begin{document}


\begin{flushleft}


{\LARGE\bf Analysing Scientific Collaborations of New Zealand Institutions using Scopus Bibliometric Data}


\vspace{1.0cm}

Samin Aref$^1$, David Friggens$^2$ and Shaun Hendy$^3$

\begin{description}

\item $^1 \;$ Department of Computer Science and Te P\={u}naha Matatini, University of Auckland, New Zealand


\item $^2 \;$ Ministry of Business Innovation \& Employment, New Zealand


\item $^3 \;$ Department of Physics and Te P\={u}naha Matatini, University of Auckland, New Zealand


\end{description}

\end{flushleft}

%  ***** ADD ENOUGH VERTICAL SPACE HERE TO ENSURE THAT THE *****
%  ***** ABSTRACT (OR MAIN TEXT) STARTS 5 CM BELOW THE TOP *****

%\vspace{0.75cm}

\noindent {\bf Abstract}. Scientific collaborations are among the main enablers of development in small national science systems. Although analysing scientific collaborations is a well-established subject in scientometrics, evaluations of collaborative activities of countries remain speculative with studies based on a limited number of fields or using data too inadequate to fully represent collaborations at a national level. This study provides a unique view on the collaborative aspect of scientific activities in New Zealand.
We perform a quantitative study based on all Scopus publications in all subjects for over 1500 New Zealand institutions over a period of 6 years to generate an extensive mapping of New Zealand scientific collaborations. The comparative results reveal the levels of collaboration between New Zealand institutions and business enterprises, government institutions, higher education providers, and private not for profit organisations in 2010-2015.
Constructing a collaboration network of institutions, we observe a power-law distribution indicating that a small number of New Zealand institutions account for a large proportion of national collaborations. Network centrality measures are deployed to identify the most influential institutions of the country in terms of scientific collaboration. We also provide comparative results on 15 universities and crown research institutes based on 27 subject classifications.
This study was based on Scopus custom data and supported by the Te Pūnaha Matatini internship program at Ministry of Business, Innovation \& Employment.

\noindent
ArXiv preprint link: https://arxiv.org/pdf/1709.02897

%\vskip 2mm

\noindent {\bf Keywords}.
Big data modelling, Scientific collaboration, Scientometrics, Network analysis, Scopus, New Zealand

\end{document}





