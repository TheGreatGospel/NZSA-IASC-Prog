\documentclass[12pt]{article}
% \documentstyle{iascars2017}

% \usepackage{iascars2017}

\pagestyle{myheadings}
\pagenumbering{arabic}
\topmargin 0pt \headheight 23pt \headsep 24.66pt
%\topmargin 11pt \headheight 12pt \headsep 13.66pt
\parindent = 3mm


\begin{document}


\begin{flushleft}


{\LARGE\bf Sparse Common Component Analysis}


\vspace{1.0cm}

Heewon Park$^1$ and Sadanori Konishi$^2$

\begin{description}

\item $^1 \;$ Faculty of Global and Science Studies, Yamaguchi University, Yamaguchi Prefecture, Japan
\item $^2 \;$ Department of Mathematics, Faculty of Science and Engineering, Chuo University, Tokyo, Japan

\end{description}

\end{flushleft}

%  ***** ADD ENOUGH VERTICAL SPACE HERE TO ENSURE THAT THE *****
%  ***** ABSTRACT (OR MAIN TEXT) STARTS 5 CM BELOW THE TOP *****

\vspace{0.75cm}

\noindent {\bf Abstract}. 
Common component analysis (CCA) was proposed by generalizing techniques for principal component analysis (PCA) for a common basic structure identification (Wang et al., 2011).
Although CCA can identify the common structure of multiple datasets, which cannot be extracted by standard PCA, the common components by CCA are estimated as linear combinations of all variables, and thus it is difficult to interpret the identified common components. 
Furthermore, the CCA procedure may be disturbed by noisy features, because CCA is based on the fully dense loadings.
To address these issues, we proposed sparse common component analysis based on $L_{1}$-type regularized regression modeling.
The proposed CCA can be performed by iterative sparse non-centered PCA for a square root of a gram matrix.
We also propose an algorithm to estimate sparse common loadings of multiple datasets in line with the algorithm of CCA.
We observe from the numerical studies that the proposed CCA can incorporate sparsity into the common loading estimation, and recover a sparse common structure efficiently in multiple dataset analysis.
\vskip 2mm

\noindent {\bf Keywords}.
Common component analysis, $L_{1}$-type regularized regression, Sparse non-centered principal component analysis

%\section{ First-level heading}
%The C98 head 1 style leaves a half-line spacing below a
%first-level heading. There should be one blank line above
%a first-level heading.
%
%\subsection { Second-level heading}
%There should also be one blank line above a second- or
%third-level heading (but no extra space below them).
%
%Do not intent the first paragraph following a heading.
%Second and subsequent paragraphs are indented by one Tab
%character (= 3 mm). If footnotes are used, they should be
%placed at the foot of the page\footnote{ Footnotes are separated
%from the text by a blank line and a printed line of length 3.5 cm.
%They should be printed in 9-point Times Roman in single line spacing.}.
%
%\subsubsection { Third-level heading}
%Please specify references using the conventions
%illustrated below. Each should begin on a new line, and
%second and subsequent lines should be on the same page
%indented by 3 mm.

\subsection*{References}

\begin{description}
\item
Wang, H., Banerjee, A. and Boley, D. (2011). 
Common component analysis for multiple covariance matrices. 
\textit{Proceedings of the 17th ACM SIGKDD International Conference on Knowledge Discovery and Data Mining}, 956--964.
\end{description}

\end{document}





