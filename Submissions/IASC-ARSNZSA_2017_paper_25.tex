\documentclass[12pt]{article}
% \documentstyle{iascars2017}

% \usepackage{iascars2017}

\pagestyle{myheadings} 
\pagenumbering{arabic}
\topmargin 0pt \headheight 23pt \headsep 24.66pt
%\topmargin 11pt \headheight 12pt \headsep 13.66pt
\parindent = 3mm 


\begin{document}


\begin{flushleft}


{\LARGE\bf R\&D policy regimes in France: New evidence from a spatio-temporal
analysis}


\vspace{1.0cm}

Benjamin Montmartin$^1$, Marcos Herrera$^2$ and Nadine Massard$^3$

\begin{description}

\item $^1 \;$ University Cote d'Azur, CNRS, GREDEG UMR 7321; Valbonne, FRANCE

\item $^2 \;$ CONICET - IELDE, National University of Salta,
Salta, Argentina

\item $^3 \;$ University Grenoble Alpes, UMR GAEL, Grenoble, France

\end{description}

\end{flushleft}

%  ***** ADD ENOUGH VERTICAL SPACE HERE TO ENSURE THAT THE *****
%  ***** ABSTRACT (OR MAIN TEXT) STARTS 5 CM BELOW THE TOP *****

\vspace{0.75cm}

\noindent {\bf Abstract}. Using a unique database containing information on the amount of R\&D
tax credits and regional, national and European subsidies received
by firms in French NUTS3 regions over the period 2001-2011, we provide
new evidence on the efficiency of R\&D policies taking into account
spatial dependency across regions. By estimating a spatial Durbin
model with regimes and fixed effects, we show that in a context of
yardstick competition between regions, national subsidies are the
only instrument that displays total leverage effect. For other instruments
internal and external effects balance each other resulting in insignificant
total effects. Structural breaks corresponding to tax credit reforms
are also revealed.

\vskip 2mm

\noindent {\bf Keywords}.
 Additionality, French policy mix, Spatial panel, Structural break

%\section{ First-level heading}
%The C98 head 1 style leaves a half-line spacing below a
%first-level heading. There should be one blank line above
%a first-level heading.
%        
%\subsection { Second-level heading}
%There should also be one blank line above a second- or
%third-level heading (but no extra space below them).
%
%Do not intent the first paragraph following a heading.
%Second and subsequent paragraphs are indented by one Tab
%character (= 3 mm). If footnotes are used, they should be
%placed at the foot of the page\footnote{ Footnotes are separated
%from the text by a blank line and a printed line of length 3.5 cm.
%They should be printed in 9-point Times Roman in single line spacing.}.
%        
%\subsubsection { Third-level heading}
%Please specify references using the conventions
%illustrated below. Each should begin on a new line, and
%second and subsequent lines should be on the same page
%indented by 3 mm.

\subsection*{References}

\begin{description}


\item
Pesaran, M. H. (2007).
A simple panel unit root test in the presence of cross-section dependence
In: \textit{Journal of
Applied Econometrics}, 
\textbf{22}, 265--312.

\item
Hendry, D. F. (1979). 
Predictive failure and econometric modelling in macroeconomics: The transactions
demand for money. 
In: \textit{P. Ormerod (Ed.), Economic Modelling: Current Issues and Problems in
Macroeconomic Modelling in the UK and the US}, 
\textbf{9}, 217--242.
Heinemann Education Books,
London.


\end{description}

\end{document}





