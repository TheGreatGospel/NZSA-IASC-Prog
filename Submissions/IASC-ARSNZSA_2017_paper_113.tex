\documentclass[12pt]{article}
% \documentstyle{iascars2017}

% \usepackage{iascars2017}

\pagestyle{myheadings} 
\pagenumbering{arabic}
\topmargin 0pt \headheight 23pt \headsep 24.66pt
%\topmargin 11pt \headheight 12pt \headsep 13.66pt
\parindent = 3mm 


\begin{document}


\begin{flushleft}


{\LARGE\bf Statistical models for the source attribution of zoonotic diseases: A study of campylobacteriosis}


\vspace{1.0cm}

Sih-Jing Liao$^1$, Jonathan Marshall$^1$, Martin L. Hazelton$^1$ and Nigel P. French$^2$

\begin{description}

\item $^1 \;$ Institute of Fundamental Sciences-Statistics, Massey University, Private Bag 11222, Palmerston North, New Zealand

\item $^2 \;$ mEpiLab, Hopkirk Research Institute, Massey University, Private Bag 11222, Palmerston North, New Zealand

\end{description}

\end{flushleft}

%  ***** ADD ENOUGH VERTICAL SPACE HERE TO ENSURE THAT THE *****
%  ***** ABSTRACT (OR MAIN TEXT) STARTS 5 CM BELOW THE TOP *****

\vspace{0.75cm}

\noindent {\bf Abstract}. Preventing and controlling zoonoses with a public health policy depends on the knowledge scientists have about the transmitted pathogens. Modelling jointly the epidemiological data and genetic information provides a methodology for tracing back the source of infection. However, this creates difficulties in assessing genetic efforts behind models of the final statistical inferences due to increased model complexity. To explore the genetic effects in the joint model, we develop a genetic free model and compare it to the joint model. We apply the two models to a recent campylobacteriosis study to estimate the attribution probability for each source. A spatial covariate is also considered in the models in order to investigate the effect of the level of rurality on the source attributions. Comparing the attributions generated by the two models, we find that: i) the genetic information integrated in the joint model gives a little more precise inference to the sparse cases observed in highly rural areas than the genetic free model; ii) on the logit scale, source attribution probabilities follow linear trends against level of rurality; and iii) poultry is the dominant source of campylobacteriosis in urban centres, whereas ruminants are the most attributable source when in rural areas.

\vskip 2mm

\noindent {\bf Keywords}.
source attribution, \textit{Campylobacter}, multinomial model, Dirichlet prior, HPD interval, DIC


%\section{ First-level heading}
%The C98 head 1 style leaves a half-line spacing below a
%first-level heading. There should be one blank line above
%a first-level heading.
%        
%\subsection { Second-level heading}
%There should also be one blank line above a second- or
%third-level heading (but no extra space below them).
%
%Do not intent the first paragraph following a heading.
%Second and subsequent paragraphs are indented by one Tab
%character (= 3 mm). If footnotes are used, they should be
%placed at the foot of the page\footnote{ Footnotes are separated
%from the text by a blank line and a printed line of length 3.5 cm.
%They should be printed in 9-point Times Roman in single line spacing.}.
%        
%\subsubsection { Third-level heading}
%Please specify references using the conventions
%illustrated below. Each should begin on a new line, and
%second and subsequent lines should be on the same page
%indented by 3 mm.

\subsection*{References}

\begin{description}

\item
Bronowski, C., James, C.E. and Winstanley, C. (2014).
Role of environmental survival in transmission of \textit{Campylobacter jejuni}.
\textit{FEMS Microbiol Lett.}, \textbf{356}(1) 8--19.

\item
Dingle, K.E., Colles, F.M., Wareing, D.R., Ure, R., Fox, A.J., Bolton, F.E., Bootsma, H.J., Willems, R.J. and Maiden, M.C. (2001).  
Multilocus sequence typing system for \textit{Campylobacter jejuni}.
\textit{J Clin Microbiol}, \textbf{39}(1):14--23.

\item
Marshall, J.C. and French, N.P. (2015).
Source attribution January to December 2014 of human \textit{Campylobacter jejuni} cases from the Manawatu.
\textit{Technical {R}eport}.

\item
Wilson, D.J., Gabriel, E., Leatherbarrow, A.J., Cheesbrough, J., Gee, S., Bolton, E., Fox, A., Fearnhead, P., Hart, C.A. and Diggle, P.J. (2008).
Tracing the source of campylobacteriosis.
\textit{PLoS Genet}, \textbf{4}(9):e1000203.

\item
Wagenaar, J.A., French, N.P. and Havelaar, A.H. (2013).
Preventing \textit{Campylobacter} at the source: why is it so difficult?
\textit{Clin Infect Dis}, \textbf{57}(11):1600--1606.

\item
Biggs, P.J., Fearnhead, P., Hotter, G., Mohan, V., Collins-Emerson, J., Kwan, E., Besser, T.E., Cookson, A., Carter, P.E. and French, N.P. (2011).
Whole-genome comparison of two \textit{Campylobacter jejuni} isolates of the same sequence type reveals multiple loci of different ancestral lineage.
\textit{PLoS One}, \textbf{6}(11):e27121.

  
\end{description}

\end{document}





