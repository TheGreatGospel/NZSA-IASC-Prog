\documentclass[12pt]{article}
% \documentstyle{iascars2017}

% \usepackage{iascars2017}

\pagestyle{myheadings} 
\pagenumbering{arabic}
\topmargin 0pt \headheight 23pt \headsep 24.66pt
%\topmargin 11pt \headheight 12pt \headsep 13.66pt
\parindent = 3mm 


\begin{document}


\begin{flushleft}


{\LARGE\bf Correlated Defaults with a Distance to Default}


\vspace{1.0cm}

Cheng-Der Fuh $^1$ and Chu-Lan Michael Kao$^2$

\begin{description}

\item $^1 \;$ Graduate Institute of Statistics, National Central University, Taoyuan City 32001, Taiwan

\item $^2 \;$ Institute of Statistics, National Chiao-Tung University, Hsinchu 30010, Taiwan

\end{description}

\end{flushleft}

%  ***** ADD ENOUGH VERTICAL SPACE HERE TO ENSURE THAT THE *****
%  ***** ABSTRACT (OR MAIN TEXT) STARTS 5 CM BELOW THE TOP *****

\vspace{0.75cm}

\noindent {\bf Abstract}. To study the feature of default clustering in credit risk management, this paper evaluates joint default probabilities and default correlation for multiple firms through the classical default barrier structure form model.  Under the commonly used factor model, by making use of a developed renewal theory, we provide a closed form approximation for the joint default probability, default correlation and expected default time, both for a specific firm as well as the first to default among a group of firms. Moreover, based on the approximated default correlation and expected default time, we propose a multi-name Distance-to-Default, which can be used to explain the feature of corporate default clustering: contagion and co-movement. Numerical studies for evaluating a multi-name Distance-to-Default are given to illustrate our model.

\vskip 2mm

\noindent {\bf Keywords}.
Correlated defaults, factor model, distance to default, renewal theory


%\section{ First-level heading}
%The C98 head 1 style leaves a half-line spacing below a
%first-level heading. There should be one blank line above
%a first-level heading.
%        
%\subsection { Second-level heading}
%There should also be one blank line above a second- or
%third-level heading (but no extra space below them).
%
%Do not intent the first paragraph following a heading.
%Second and subsequent paragraphs are indented by one Tab
%character (= 3 mm). If footnotes are used, they should be
%placed at the foot of the page\footnote{ Footnotes are separated
%from the text by a blank line and a printed line of length 3.5 cm.
%They should be printed in 9-point Times Roman in single line spacing.}.
%        
%\subsubsection { Third-level heading}
%Please specify references using the conventions
%illustrated below. Each should begin on a new line, and
%second and subsequent lines should be on the same page
%indented by 3 mm.

%\subsection*{References}
%
%\begin{description}
%
%\item
%Barnett, J.A., Payne, R.W. and Yarrow, D. (1990).
%\textit{Yeasts: Characteristics and identification: Second Edition.}
%Cambridge: Cambridge University Press.
%
%\item
%(ed.) Barnett, V., Payne, R. and Steiner, R. (1995).
%\textit{Agricultural Sustainability: Economic, Environmental and
%Statistical Considerations}. Chichester: Wiley.
%
%\item
%Payne, R.W. (1997).
%\textit{Algorithm AS314 Inversion of matrices Statistics},
%\textbf{46}, 295--298.
%
%\item
%Payne, R.W. and Welham, S.J. (1990).
%A comparison of algorithms for combination of information in generally
%balanced designs.
%In: \textit{COMPSTAT90 Proceedings in Computational Statistics}, 297--302.
%Heidelberg: Physica-Verlag.
%
%\end{description}

\end{document}





