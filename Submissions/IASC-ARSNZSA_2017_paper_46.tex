\documentclass[12pt]{article}
% \documentstyle{iascars2017}

% \usepackage{iascars2017}

\pagestyle{myheadings} 
\pagenumbering{arabic}
\topmargin 0pt \headheight 23pt \headsep 24.66pt
%\topmargin 11pt \headheight 12pt \headsep 13.66pt
\parindent = 3mm 


\begin{document}


\begin{flushleft}


{\LARGE\bf Intensity Estimation of Spatial Point Processes Based on Area-Aggregated Data }


\vspace{1.0cm}

Hsin-Cheng Huang and Chi-Wei Lai

\begin{description}

\item Institute of Statistical Science, Academia Sinica, Taipei 11529, Taiwan

\end{description}

\end{flushleft}

%  ***** ADD ENOUGH VERTICAL SPACE HERE TO ENSURE THAT THE *****
%  ***** ABSTRACT (OR MAIN TEXT) STARTS 5 CM BELOW THE TOP *****

\vspace{0.75cm}

\noindent {\bf Abstract}. We consider estimation of intensity function for spatial point processes based on area-aggregated data.
A standard approach for estimating the intensity function for a spatial point pattern is to use a kernel estimator.
However, when data are only available in a spatially aggregated form with the numbers of events available in geographical subregions,
traditional methods developed for individual-level event data become infeasible. In this research, a kernel-based method will be proposed to produce a smooth intensity function based on aggregated count data. Some numerical examples will be provided to demonstrate the effectiveness of the proposed method.
\vskip 2mm

\noindent {\bf Keywords}.
Area censoring, inhomogeneous spatial point processes, kernel density estimation

\end{document}





