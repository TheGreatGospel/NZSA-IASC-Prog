\documentclass[12pt]{article}
% \documentstyle{iascars2017}

% \usepackage{iascars2017}

\pagestyle{myheadings} 
\pagenumbering{arabic}
\topmargin 0pt \headheight 23pt \headsep 24.66pt
%\topmargin 11pt \headheight 12pt \headsep 13.66pt
\parindent = 3mm 


\begin{document}


\begin{flushleft}


{\LARGE\bf 
Regularized Noise-Reduction Methodology for High-Dimensional Data
}


\vspace{1.0cm}

Kazuyoshi Yata$^1$ and Makoto Aoshima$^1$

\begin{description}

\item $^1 \;$
Institute of Mathematics, University of Tsukuba, 
Ibaraki 305-8571, Japan

\end{description}

\end{flushleft}

%  ***** ADD ENOUGH VERTICAL SPACE HERE TO ENSURE THAT THE *****
%  ***** ABSTRACT (OR MAIN TEXT) STARTS 5 CM BELOW THE TOP *****

\vspace{0.75cm}

\noindent {\bf Abstract}. 
In this talk, we consider principal component analysis (PCA) methods in high-dimensional settings. We first consider asymptotic properties of the conventional estimator of eigenvalues. We show that the estimator is affected by the high-dimensional noise structure directly, so that it becomes inconsistent. In order to overcome such difficulties in a high-dimensional situation, Yata and Aoshima (2012) developed a new PCA method called the noise-reduction (NR) methodology. We show that the NR method can enjoy consistency properties not only for eigenvalues but also for PC directions in high-dimensional settings. The estimator of the PC directions by the NR method has a consistency property in terms of an inner product. However, it does not hold a consistency property in terms of the Euclid norm. With the help of a thresholding method, we modify the estimator and propose a regularized NR method. We show that it holds the consistency property of the Euclid norm. Finally, we check the performance of the new NR method by using microarray data sets. 
\vskip 2mm

\noindent {\bf Keywords}.
eigenstructure, large $p$ small $n$, PCA, spiked model

%\section{ First-level heading}
%The C98 head 1 style leaves a half-line spacing below a
%first-level heading. There should be one blank line above
%a first-level heading.
%        
%\subsection { Second-level heading}
%There should also be one blank line above a second- or
%third-level heading (but no extra space below them).
%
%Do not intent the first paragraph following a heading.
%Second and subsequent paragraphs are indented by one Tab
%character (= 3 mm). If footnotes are used, they should be
%placed at the foot of the page\footnote{ Footnotes are separated
%from the text by a blank line and a printed line of length 3.5 cm.
%They should be printed in 9-point Times Roman in single line spacing.}.
%        
%\subsubsection { Third-level heading}
%Please specify references using the conventions
%illustrated below. Each should begin on a new line, and
%second and subsequent lines should be on the same page
%indented by 3 mm.

\subsection*{References}

\begin{description}


\item
Yata, K. and Aoshima. M. (2012). 
Effective PCA for high-dimension, low-sample-size data with noise reduction via geometric representations. 
\textit{Journal of Multivariate Analysis},
\textbf{105}, 193--215. 

\end{description}

\end{document}





