\documentclass[12pt]{article}
\usepackage[textwidth=20cm]{geometry}
\usepackage[T1]{fontenc}
\usepackage[utf8]{inputenc}
\usepackage{authblk}

\title{\textbf{Threshold Determination for the Meteorological Data Quality Control in Korea}}
\author[1*]{Yung-Seop Lee}
\author[2]{Hee-Kyung Kim}
\author[3]{Myungjin Hyun}
\affil[1*]{Department of Statistics, Dongguk University, Seoul, 04620, Korea, yung@dongguk.edu}
\affil[2]{Department of Statistics, Dongguk University, Seoul, 04620, Korea}
\affil[3]{KMA National Climate Data Center, Seoul, 07062, Korea}
\date{}

\renewcommand\Authands{ and }

\begin{document}
  \maketitle
  
  \setlength{\parindent}{0cm}
  
  \textbf{Abstract:} The raw meteorological data need to be cleaned since they are from the diverse sources such as ASOS(Automated Synoptic Observing System) and AWS(Automatic Weather Station). The meteorological data in South Korea is observed from about 100 ASOS and 500 AWS. In order to produce the high qualified meteorological data, several data quality control algorithms are applied. In this study, cluster analysis for almost 600 meteorological sites is applied depending on their climatic characteristics. After clustering, we propose the several threshold algorithms in the given cluster. The proposed threshold values for data quality control algorithms will be adequate to Korea climate condition by cluster and month. Thresholds of QC algorithms, which are step test, persistence test and climate range test, are determined. Through these algorithms and threshold, the qualified meteorological data can be produced for the improved forecast accuracy.
  \\
  \\
  \textbf{Key Words:} meteorological data quality control, threshold values, cluster analysis, step test, persistence test, climate range test.
  
\end{document}