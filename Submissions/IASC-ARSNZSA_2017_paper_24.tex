\documentclass[12pt]{article}
% \documentstyle{iascars2017}

% \usepackage{iascars2017}

\pagestyle{myheadings} 
\pagenumbering{arabic}
\topmargin 0pt \headheight 23pt \headsep 24.66pt
%\topmargin 11pt \headheight 12pt \headsep 13.66pt
\parindent = 3mm 


\begin{document}


\begin{flushleft}


{\LARGE\bf Identifying Clusters Of Patients With Diabetes Using a Markov Birth-Death Process}


\vspace{1.0cm}

Mugdha Manda$^1$, Thomas Lumley$^1$ and Susan Wells$^2$

\begin{description}

\item $^1 \;$ The Department of Statistics, The University of Auckland, Auckland 1010, New Zealand

\item $^2 \;$ Section of Epidemiology and Biostatistics, Faculty of Medical and Health Sciences, The University of Auckland, Auckland 1072, New Zealand

\end{description}

\end{flushleft}

\vspace{1.2cm}

\noindent {\bf Abstract}. Estimating disease trajectories has increasingly become more essential to clinical practitioners to administer effective treatment to their patients. A part of describing disease trajectories involves taking patients' medical histories and sociodemographic factors into account and grouping them into similar groups, or clusters. Advances in computerised patient databases have paved a way for identifying such trajectories in patients by recording a patient's medical history over a long period of time (longitudinal data): we studied data from the PREDICT-CVD dataset, a national primary-care cohort from which people with diabetes from 2002-2015 were identified through routine clinical practice. We fitted a Bayesian hierarchical linear model with latent clusters to the repeated measurements of HbA$_1c$ and eGFR, using the Markov birth-death process proposed by Stephens (2000) to handle the changes in dimensionality as clusters were added or removed.

\vskip 2mm

\noindent {\bf Keywords}.
Diabetes management,
longitudinal data,
Markov chain Monte Carlo,
birth-death process,
mixture model,
Bayesian analysis,
latent clusters,
hierarchical models,
primary care,
clinical practice



\subsection*{References}

\begin{description}

\item
Stephens, M. (2000). Bayesian Analysis of Mixture Models with an Unknown Number of Components - An Alternative to Reversible Jump Methods. 
In: \textit{The Annals of Statistics}, 28(1), 40-74. 

\end{description}

\end{document}





