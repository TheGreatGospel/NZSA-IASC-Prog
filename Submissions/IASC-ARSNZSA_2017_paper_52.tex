\documentclass[12pt]{article}
% \documentstyle{iascars2017}

% \usepackage{iascars2017}

\pagestyle{myheadings} 
\pagenumbering{arabic}
\topmargin 0pt \headheight 23pt \headsep 24.66pt
%\topmargin 11pt \headheight 12pt \headsep 13.66pt
\parindent = 3mm 


\begin{document}


\begin{flushleft}


{\LARGE\bf Smoothing Nonparametric Regression under Shape Restrictions}


\vspace{1.0cm}

Hongbin Guo and Yong Wang

\begin{description}

\item Department of Statistics, University of Auckland


\end{description}

\end{flushleft}

%  ***** ADD ENOUGH VERTICAL SPACE HERE TO ENSURE THAT THE *****
%  ***** ABSTRACT (OR MAIN TEXT) STARTS 5 CM BELOW THE TOP *****

\vspace{0.75cm}

\noindent {\bf Abstract}. Shape-restricted regression, in particular under isotonicity and convexity(concavity) constraints, has many practical applications. Traditional nonparametric methods to the problem using least squares or maximum likelihood result in discrete step functions or nonsmooth piecewise linear functions, which are unsatisfactory both predictively and visually. In this talk, we describe a new, smooth, nonparametric estimator under the above-mentioned shape restrictions. In particular, the discrete measures that are inherent in the previous estimators are replaced with continuous ones. A new algorithm that can rapidly find the corresponding estimate will also be presented. Numerical studies show that the new estimator outperforms major existing methods in almost all cases.

\vskip 2mm

\noindent {\bf Keywords}.
Nonparametric regression, smooth, shape restriction, convex, monotonic


%\section{ First-level heading}
%The C98 head 1 style leaves a half-line spacing below a
%first-level heading. There should be one blank line above
%a first-level heading.
%        
%\subsection { Second-level heading}
%There should also be one blank line above a second- or
%third-level heading (but no extra space below them).
%
%Do not intent the first paragraph following a heading.
%Second and subsequent paragraphs are indented by one Tab
%character (= 3 mm). If footnotes are used, they should be
%placed at the foot of the page\footnote{ Footnotes are separated
%from the text by a blank line and a printed line of length 3.5 cm.
%They should be printed in 9-point Times Roman in single line spacing.}.
%        
%\subsubsection { Third-level heading}
%Please specify references using the conventions
%illustrated below. Each should begin on a new line, and
%second and subsequent lines should be on the same page
%indented by 3 mm.

\subsection*{References}

\begin{description}

\item
 Groeneboom, P., Jongbloed, G. and Wellner, A. (2001). Estimation of a Convex Function: Characterizations and Asymptotic Theory. \textit{Ann. Statist}. \textbf{29}(6), 1653--1698. 

\item
Wang, Y. (2007). On fast computation of the non-parametric maximum likelihood estimate of a mixing distribution. \textit{Journal of the Royal Statistical Society. Series B: Statistical Methodology}, \textbf{69} (2), 185--198. 


\item
Meyer, M.(2008). Inference using shape-restricted regression splines. 
\textit{Ann. Appl. Stat.} \textbf{2}(3), 1013--1033.
\end{description}

\end{document}
