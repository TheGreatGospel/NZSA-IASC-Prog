\documentclass[12pt]{article}
% \documentstyle{iascars2017}

% \usepackage{iascars2017}

\pagestyle{myheadings} 
\pagenumbering{arabic}
\topmargin 0pt \headheight 23pt \headsep 24.66pt
%\topmargin 11pt \headheight 12pt \headsep 13.66pt
\parindent = 3mm 


\begin{document}


\begin{flushleft}


{\LARGE\bf Lattice Polytope Samplers}


\vspace{1.0cm}

Martin L.~Hazelton$^1$

\begin{description}

\item $^1 \;$ Institute of Fundamental Sciences, Massey University, Palmerston North 4410, New Zealand



\end{description}

\end{flushleft}

%  ***** ADD ENOUGH VERTICAL SPACE HERE TO ENSURE THAT THE *****
%  ***** ABSTRACT (OR MAIN TEXT) STARTS 5 CM BELOW THE TOP *****

\vspace{0.75cm}

\noindent {\bf Abstract}. 
Statistical inverse problems occur when we wish to learn about some random process that is
observed only indirectly. Inference in such situations typically involves sampling possible values
for the latent variables of interest conditional on the indirect observations. This talk is concerned
with inverse problems for count data, for which the latent variables are constrained to lie on the
integer lattice within a convex polytope (a bounded multidimensional polyhedron). An illustrative
example arises in transport engineering where we observe vehicle counts entering or leaving each
zone of the network, then want to sample possible interzonal patterns of traffic flow consistent with
those entry/exit counts. Other problems with this structure arise when conducting exact inference
for contingency tables, and when analysing capture-recapture data in ecology.

In principle such sampling can be conducted using Markov chain Monte Carlo methods through a
random walk on the lattice polytope, but it is challenging to design algorithms for doing so that are
both computationally efficient and have guaranteed theoretical properties. The seminal work of 
Diaconis and Sturmfels (1998) on Markov bases addresses some of the theoretical issues, but has 
significant practical limitations. In this talk I shall discuss some preliminary findings based on 
a more geometric approach to sampler design.


\vskip 2mm

\noindent {\bf Keywords}.
lattice bases, Markov bases, MCMC, statistical linear inverse problem

%\section{ First-level heading}
%The C98 head 1 style leaves a half-line spacing below a
%first-level heading. There should be one blank line above
%a first-level heading.
%        
%\subsection { Second-level heading}
%There should also be one blank line above a second- or
%third-level heading (but no extra space below them).
%
%Do not intent the first paragraph following a heading.
%Second and subsequent paragraphs are indented by one Tab
%character (= 3 mm). If footnotes are used, they should be
%placed at the foot of the page\footnote{ Footnotes are separated
%from the text by a blank line and a printed line of length 3.5 cm.
%They should be printed in 9-point Times Roman in single line spacing.}.
%        
%\subsubsection { Third-level heading}
%Please specify references using the conventions
%illustrated below. Each should begin on a new line, and
%second and subsequent lines should be on the same page
%indented by 3 mm.

\subsection*{References}

\begin{description}
\item
Diaconis, P., and Sturmfels, B. (1998). Algebraic algorithms for sampling from conditional distributions.
\textit{The Annals of Statistics} {\bf 26}, 363-397.
\end{description}

\end{document}





