\documentclass[12pt]{article}
% \documentstyle{iascars2017}

% \usepackage{iascars2017}

\pagestyle{myheadings} 
\pagenumbering{arabic}
\topmargin 0pt \headheight 23pt \headsep 24.66pt
%\topmargin 11pt \headheight 12pt \headsep 13.66pt
\parindent = 3mm 


\begin{document}


\begin{flushleft}


{\Large\bf High Dimensional Asymptotics for the Naive Canonical Correlation Coefficient}


\vspace{1.0cm}

Mitsuru Tamatani$^1$ and Kanta Naito$^2$

\begin{description}

\item $^1 \;$ Faculty of Culture and Information Science, Doshisha University, Japan

\item $^2 \;$ Graduate school of science and engineering, Shimane University, Japan

\end{description}

\end{flushleft}

%  ***** ADD ENOUGH VERTICAL SPACE HERE TO ENSURE THAT THE *****
%  ***** ABSTRACT (OR MAIN TEXT) STARTS 5 CM BELOW THE TOP *****

\vspace{0.75cm}

\noindent {\bf Abstract}. 
In this talk we investigate the asymptotic behavior of the estimated naive canonical correlation coefficient under the normality assumption and High Dimension Low Sample Size (HDLSS) settings.
In general, canonical correlation matrix is associated with canonical correlation analysis which is useful in studying the relationship between two sets of variables.
However, in HDLSS settings, the within-class sample covariance matrix $\hat{\Sigma}$ is singular, because the rank of $\hat{\Sigma}$ is much less than the number of dimension.
To avoid the singularity of $\hat{\Sigma}$ in HDLSS settings,
we utilize the naive canonical correlation matrix with replacing sample covariance matrix by its diagonal
part only. 
We derive the asymptotic normality of the estimated naive canonical correlation coefficient, and compare the results of our numerical studies to the theoretical asymptotic results.

\vskip 2mm

\noindent {\bf Keywords}.
High dimension low sample size, Naive canonical correlation coefficient, Asymptotic normality


%\section{ First-level heading}
%The C98 head 1 style leaves a half-line spacing below a
%first-level heading. There should be one blank line above
%a first-level heading.
%        
%\subsection { Second-level heading}
%There should also be one blank line above a second- or
%third-level heading (but no extra space below them).
%
%Do not intent the first paragraph following a heading.
%Second and subsequent paragraphs are indented by one Tab
%character (= 3 mm). If footnotes are used, they should be
%placed at the foot of the page\footnote{ Footnotes are separated
%from the text by a blank line and a printed line of length 3.5 cm.
%They should be printed in 9-point Times Roman in single line spacing.}.
%        
%\subsubsection { Third-level heading}
%Please specify references using the conventions
%illustrated below. Each should begin on a new line, and
%second and subsequent lines should be on the same page
%indented by 3 mm.

\subsection*{References}

\begin{description}

\item
Tamatani, M., Koch, I. and Naito, K. (2012).
\textit{Journal of Multivariate Analysis},
\textbf{111}, 350--367.

\item
Srivastava, M. S. (2011).
\textit{Journal of Multivariate Analysis},
\textbf{102}, 1190--1103.

\item
Fan, J. and Fan, Y. (2008).
\textit{The Annals of Statistics},
\textbf{36}, 2605--2637.


\end{description}

\end{document}





