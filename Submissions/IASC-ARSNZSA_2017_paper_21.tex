\documentclass[12pt]{article}
% \documentstyle{iascars2017}

% \usepackage{iascars2017}

\pagestyle{myheadings} 
\pagenumbering{arabic}
\topmargin 0pt \headheight 23pt \headsep 24.66pt
%\topmargin 11pt \headheight 12pt \headsep 13.66pt
\parindent = 3mm 


\begin{document}


\begin{flushleft}


{\LARGE\bf Clustering using Nonparametric Mixtures and Mode Identification}


\vspace{1.0cm}

Shengwei Hu$^1$ and Yong Wang$^2$

\begin{description}

\item $^1 \;$ Department of Statistics, the University of Auckland,
New Zealand

\item $^2 \;$ Department of Statistics, the University of Auckland,
New Zealand

\end{description}

\end{flushleft}

%  ***** ADD ENOUGH VERTICAL SPACE HERE TO ENSURE THAT THE *****
%  ***** ABSTRACT (OR MAIN TEXT) STARTS 5 CM BELOW THE TOP *****

\vspace{0.75cm}

\noindent {\bf Abstract}. Clustering aims to partition a set of observations into a proper number of clusters with similar objects allocated to the same group. Current partitioning methods mainly include those based on some measure of distance or probability distribution. Here we propose a mode-based clustering methodology motivated via density estimation and mode identification procedures. The idea is to estimate the data-generating probability distribution using a nonparametric density estimator and then locate the modes of the density obtained. In the nonparametric mixture models, each mode and the observations ascend to it correspond to a single cluster. Thus, the problem of determining the number of clusters can be recast as a mode merging problem. A criterion of measuring the separability between modes is also addressed in this work. The most similar modes would be merged sequentially until the optimal number of clusters is reached. The performance of the proposed method is investigated on both simulated and real datasets.


\vskip 2mm

\noindent {\bf Keywords}.
Clustering, Nonparametric mixtures, Mode identification


%\section{ First-level heading}
%The C98 head 1 style leaves a half-line spacing below a
%first-level heading. There should be one blank line above
%a first-level heading.
%        
%\subsection { Second-level heading}
%There should also be one blank line above a second- or
%third-level heading (but no extra space below them).
%
%Do not intent the first paragraph following a heading.
%Second and subsequent paragraphs are indented by one Tab
%character (= 3 mm). If footnotes are used, they should be
%placed at the foot of the page\footnote{ Footnotes are separated
%from the text by a blank line and a printed line of length 3.5 cm.
%They should be printed in 9-point Times Roman in single line spacing.}.
%        
%\subsubsection { Third-level heading}
%Please specify references using the conventions
%illustrated below. Each should begin on a new line, and
%second and subsequent lines should be on the same page
%indented by 3 mm.

\subsection*{References}

\begin{description}

\item
Wang, X. and Wang, Y.:
\textit{Nonparametric multivariate density estimation using mixtures}.
Stat. Comput. \textbf{25}, 349–-364 (2015).

\item
Li, J., Ray S. and Lindsay B.G.:
\textit{A nonparametric statistical approach to clustering via mode identification}.
Journal of Machine Learning Research. \textbf{8}, 1687–-1723 (2007).



\end{description}

\end{document}





