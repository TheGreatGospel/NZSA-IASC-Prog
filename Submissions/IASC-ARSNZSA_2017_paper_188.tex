\documentclass[11pt,]{article}
\usepackage[]{mathpazo}
\usepackage[T1]{fontenc}
\usepackage[utf8]{inputenc}
\usepackage{microtype}
\usepackage[margin=1in]{geometry}
\usepackage{parskip}

\title{Probabilistic outlier detection and visualization of smart meter data}
\author{Rob J Hyndman}
\date{Monash University, Clayton VIC 3800, Australia}

\begin{document}
\maketitle

It is always a good idea to plot your data before fitting any models,
making any predictions, or drawing any conclusions. But how do you
actually plot data on thousands of smart meters, each comprising
thousands of observations over time? We cannot simply produce time plots
of the demand recorded at each meter, due to the sheer volume of data
involved.

I will propose an approach in which each long series of demand data is
converted to a single two-dimensional point that can be plotted in a
simple scatterplot. In that way, all the meters can be seen in the
scatterplot; so outliers can be detected, clustering can be observed,
and any other interesting structure can be examined. To illustrate, I
will use data collected during a smart metering trial conducted by the
Commission for Energy Regulation (CER) in Ireland.

First we estimate the demand percentiles for each half hour of the week,
giving us 336 probability distributions per household. Then, we compute
the distances between pairs of households using the sum of
Jensen--Shannon distances.

From these pairwise distances, we can compute a measure of the
``typicality'' of a specific household, by seeing how many similar
houses are nearby. If there are many households with similar probability
distributions, the typicality measure will be high. But if there are few
similar households, the typicality measure will be low. This gives us a
way of finding anomalies in the data set --- they are the smart meters
corresponding to the least typical households.

The pairwise distances between households can also be used to create a
plot of all households together. Each of the household distributions can
be thought of as a vector in \(K\)-dimensional space where
\(K=7\times48\times99 = 33,264\). To easily visualize these, we need to
project them onto a two-dimensional space. I propose using Laplacian
eigenmaps which attempt to preserve the smallest distances --- so the
most similar points in \(K\)-dimensional space are as close as possible
in the two-dimensional space.

This way of plotting the data easily allows us to see the anomalies, to
identify any clusters of observations in the data, and to examine any
other structure that might exist.

\end{document}
