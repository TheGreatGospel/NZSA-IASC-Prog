\documentclass[12pt]{article}
% \documentstyle{iascars2017}

% \usepackage{iascars2017}

\pagestyle{myheadings} 
\pagenumbering{arabic}
\topmargin 0pt \headheight 23pt \headsep 24.66pt
%\topmargin 11pt \headheight 12pt \headsep 13.66pt
\parindent = 3mm 


\begin{document}


\begin{flushleft}


{\LARGE\bf 
Modeling of Document Abstraction Using Association Rule Based Characterization
}


\vspace{1.0cm}

Ken Nittono

\begin{description}

\item Department of Markets and Management, Hosei University, 102-8160 Tokyo, Japan

\end{description}

\end{flushleft}

%  ***** ADD ENOUGH VERTICAL SPACE HERE TO ENSURE THAT THE *****
%  ***** ABSTRACT (OR MAIN TEXT) STARTS 5 CM BELOW THE TOP *****

\vspace{0.75cm}

\noindent {\bf Abstract}. The importance of systems enabling us to extract useful information from enormous text data produced every day along with our social activities in organizations or on the internet and utilize the information immediately and efficiently have been increasing.
%===
In this research, an analyzing method which extracts essential parts from a huge document set utilizing association rule analysis as a data mining method is introduced.
%===
The method detects typical combinations of terms involved in contexts and regards them as the characterization of text data and it is also combined with information retrieval methods for the sake of further selection as some parts of the essential contexts.
%===
This method is considered to enhance its ability of detection for particular contexts that contain some topics and include moderately distributed terminologies.
%===
And implementation of the system is discussed in order for utilizing the abstracted documents efficiently as some sort of knowledge such as collective intelligence.
%===
An approach for linkage with R is also mentioned in the phase of the implementation of the model.



\vskip 2mm

\noindent {\bf Keywords}.
Association rule, Text mining, Big data, Information retrieval


%\section{ First-level heading}
%The C98 head 1 style leaves a half-line spacing below a
%first-level heading. There should be one blank line above
%a first-level heading.
%        
%\subsection { Second-level heading}
%There should also be one blank line above a second- or
%third-level heading (but no extra space below them).
%
%Do not intent the first paragraph following a heading.
%Second and subsequent paragraphs are indented by one Tab
%character (= 3 mm). If footnotes are used, they should be
%placed at the foot of the page\footnote{ Footnotes are separated
%from the text by a blank line and a printed line of length 3.5 cm.
%They should be printed in 9-point Times Roman in single line spacing.}.
%        
%\subsubsection { Third-level heading}
%Please specify references using the conventions
%illustrated below. Each should begin on a new line, and
%second and subsequent lines should be on the same page
%indented by 3 mm.

\subsection*{References}

\begin{description}

\item
Agrawal, R. Imielinski, T. and Swami, A. (1993).
\textit{Mining association rules between sets of items in large databases},
Proceedings of the ACM SIGMOD Washington, D.C, 207--216.

\item
Nittono, K. (2013).
\textit{Association rule generation and mining approach to concept space for collective documents},
Proceedings of the 59th World Statistics Congress of the International Statistical Institute, pp. 5515--5520.

\end{description}

\end{document}
