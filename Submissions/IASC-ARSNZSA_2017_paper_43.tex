\documentclass[12pt]{article}
% \documentstyle{iascars2017}

% \usepackage{iascars2017}

\pagestyle{myheadings} 
\pagenumbering{arabic}
\topmargin 0pt \headheight 23pt \headsep 24.66pt
%\topmargin 11pt \headheight 12pt \headsep 13.66pt
\parindent = 3mm 


\begin{document}


\begin{flushleft}


{\LARGE\bf Dissimilarities between groups of data}


\vspace{1.0cm}

Nobuo SHIMIZU$^1$, Junji NAKANO$^1$ and Yoshikazu YAMAMOTO$^2$

\begin{description}

\item $^1 \;$ The Institute of Statistical Mathematics,
10-3 Midori-cho, Tachikawa, Tokyo 190-8562, Japan

\item $^2 \;$ Tokushima Bunri University,
1314-1 Shido, Sanuki-city, Kagawa 769-2193, Japan

\end{description}

\end{flushleft}

%  ***** ADD ENOUGH VERTICAL SPACE HERE TO ENSURE THAT THE *****
%  ***** ABSTRACT (OR MAIN TEXT) STARTS 5 CM BELOW THE TOP *****

\vspace{0.75cm}

\noindent {\bf Abstract}. 
We often have ``big data'' expressed by both continuous real variables and categorical variables. 
When their sizes are huge, it is almost impossible to see and check each individual data. 
Then we divide them into small number of groups which have clear domain meanings. 
We express each group by using information up to second order moments. 
For example, means, variances and covariances are used to summarize many continuous real variables, 
and a Burt matrix which consists of contingency tables by pairs of categorical variables are used to summarize many categorical variables. 
We call such a set of descriptive statistics ``aggregated symbolic data (ASD)''.

We here propose dissimilarities between two ASDs by utilizing pseudo-likelihood ratio test statistic and chi-squared test statistic. 
Former one is theoretically derived and the latter one is heuristically given. 
We adopt two dissimilarities for clustering districts in Tokyo by ASD derived from huge real estate data.

\vskip 2mm

\noindent {\bf Keywords}.
Aggregated symbolic data, Chi-squared test statistic, clustering, pseudo-likelihood ratio test statistic


%\section{ First-level heading}
%The C98 head 1 style leaves a half-line spacing below a
%first-level heading. There should be one blank line above
%a first-level heading.
%        
%\subsection { Second-level heading}
%There should also be one blank line above a second- or
%third-level heading (but no extra space below them).
%
%Do not intent the first paragraph following a heading.
%Second and subsequent paragraphs are indented by one Tab
%character (= 3 mm). If footnotes are used, they should be
%placed at the foot of the page\footnote{ Footnotes are separated
%from the text by a blank line and a printed line of length 3.5 cm.
%They should be printed in 9-point Times Roman in single line spacing.}.
%        
%\subsubsection { Third-level heading}
%Please specify references using the conventions
%illustrated below. Each should begin on a new line, and
%second and subsequent lines should be on the same page
%indented by 3 mm.

%\subsection*{References}
%
%\begin{description}
%
%\item
%Barnett, J.A., Payne, R.W. and Yarrow, D. (1990).
%\textit{Yeasts: Characteristics and identification: Second Edition.}
%Cambridge: Cambridge University Press.
%
%\item
%(ed.) Barnett, V., Payne, R. and Steiner, R. (1995).
%\textit{Agricultural Sustainability: Economic, Environmental and
%Statistical Considerations}. Chichester: Wiley.
%
%\item
%Payne, R.W. (1997).
%\textit{Algorithm AS314 Inversion of matrices Statistics},
%\textbf{46}, 295--298.
%
%\item
%Payne, R.W. and Welham, S.J. (1990).
%A comparison of algorithms for combination of information in generally
%balanced designs.
%In: \textit{COMPSTAT90 Proceedings in Computational Statistics}, 297--302.
%Heidelberg: Physica-Verlag.
%
%\end{description}

\end{document}





