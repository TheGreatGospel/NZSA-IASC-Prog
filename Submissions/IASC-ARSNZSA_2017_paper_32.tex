\documentclass[12pt]{article}
% \documentstyle{iascars2017}

% \usepackage{iascars2017}

%\pagestyle{myheadings} 
%\pagenumbering{arabic}
%\topmargin 0pt \headheight 23pt \headsep 24.66pt
%\topmargin 11pt \headheight 12pt \headsep 13.66pt
\parindent = 3mm 

\newcommand{\pkg}[1]{{\bf #1}}
\usepackage{natbib}
\setlength{\bibsep}{0.0pt}

\begin{document}


\begin{flushleft}


{\LARGE\bf gridSVG: Then and Now}

\vspace{.1in}
Paul Murrell $^1$
\begin{description}
\item $^1 $ Department of Statistics, The University of Auckland,
Auckland, New Zealand
\end{description}

\end{flushleft}

%  ***** ADD ENOUGH VERTICAL SPACE HERE TO ENSURE THAT THE *****
%  ***** ABSTRACT (OR MAIN TEXT) STARTS 5 CM BELOW THE TOP *****

%\vspace{0.75cm}

\noindent {\bf Abstract}. 
The \pkg{gridSVG} package\cite{RJ-2014-013}
 was first developed in 2003 to experiment
with features of the SVG format
that were not available through a normal R graphics device\cite{R}, 
such as hyperlinks and animation.  A number of different 
R packages\cite{rsvgtipsdevice,cairo,svglite,svgannotation}
have been developed since then to allow the generation of SVG output from
R, but \pkg{gridSVG} has remained unique in its focus on generating 
structured and labelled SVG output.  The reason for that was to 
maximise support for
customisation and reuse, particularly unforseen reuse, of the SVG
output.  Unfortunately, there were two major problems:  killer examples of
customisation and reuse failed to materialise;  and the production of SVG with 
\pkg{gridSVG} was painfully slow.  In brief, \pkg{gridSVG} was
a (sluggish) solution waiting for a problem.  This talk charts some
of the developments over time that have seen \pkg{gridSVG}'s
patient wait for relevance ultimately rewarded and its 
desperate need for speed finally satisfied.

\vskip 2mm

\noindent {\bf Keywords}.
R, statistical graphics, SVG, accessibility

%\section{ First-level heading}
%The C98 head 1 style leaves a half-line spacing below a
%first-level heading. There should be one blank line above
%a first-level heading.
%        
%\subsection { Second-level heading}
%There should also be one blank line above a second- or
%third-level heading (but no extra space below them).
%
%Do not intent the first paragraph following a heading.
%Second and subsequent paragraphs are indented by one Tab
%character (= 3 mm). If footnotes are used, they should be
%placed at the foot of the page\footnote{ Footnotes are separated
%from the text by a blank line and a printed line of length 3.5 cm.
%They should be printed in 9-point Times Roman in single line spacing.}.
%        
%\subsubsection { Third-level heading}
%Please specify references using the conventions
%illustrated below. Each should begin on a new line, and
%second and subsequent lines should be on the same page
%indented by 3 mm.

\renewcommand{\section}[2]{}%
\bibliographystyle{plain}
\bibliography{murrell}


\end{document}





