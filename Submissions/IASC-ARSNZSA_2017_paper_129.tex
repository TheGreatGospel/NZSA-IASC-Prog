\documentclass[12pt]{article}
% \documentstyle{iascars2017}

% \usepackage{iascars2017}

\pagestyle{myheadings} 
\pagenumbering{arabic}
\topmargin 0pt \headheight 23pt \headsep 24.66pt
%\topmargin 11pt \headheight 12pt \headsep 13.66pt
\parindent = 3mm 


\begin{document}


\begin{flushleft}


{\LARGE\bf Separation of symmetry for square contingency tables with ordinal categories}


\vspace{1.0cm}

Kouji Tahata$^1$
\begin{description}

\item $^1 \;$ Department of Information Sciences, Tokyo University of Science,
2641 Yamazaki, Noda-shi, Chiba-ken, 2788510, JAPAN

\end{description}

\end{flushleft}

%  ***** ADD ENOUGH VERTICAL SPACE HERE TO ENSURE THAT THE *****
%  ***** ABSTRACT (OR MAIN TEXT) STARTS 5 CM BELOW THE TOP *****

\vspace{0.75cm}

\noindent {\bf Abstract}. Symmetry and asymmetry models are used to analyze a {\it square} contingency table with ordinal categories.
Caussinus (1966) pointed out that the symmetry model, which indicates the structure of symmetry for cell probabilities, could be separated into the structure of symmetry for odds-ratios and that of symmetry for marginal distributions.
This result provides the reason for poor fit of the symmetry model when it occurs for a real dataset.
Also, other separations of the symmetry model have been given.
For example, Kateri and Agresti (2007), and Saigusa et al. (2015).
In this paper, we consider the separation of symmetry by using the generalized asymmetry models.
A theorem which the likelihood ratio statistic for testing goodness of fit of the symmetry model is asymptotically equivalent to the sum of those for testing the generalized asymmetry model and the moment equality model under some conditions is given.
A simulation study is presented.

\vskip 2mm

\noindent {\bf Keywords}.
$f$-divergence, moment equality, orthogonality, quasi-symmetry


%\section{ First-level heading}
%The C98 head 1 style leaves a half-line spacing below a
%first-level heading. There should be one blank line above
%a first-level heading.
%        
%\subsection { Second-level heading}
%There should also be one blank line above a second- or
%third-level heading (but no extra space below them).
%
%Do not intent the first paragraph following a heading.
%Second and subsequent paragraphs are indented by one Tab
%character (= 3 mm). If footnotes are used, they should be
%placed at the foot of the page\footnote{ Footnotes are separated
%from the text by a blank line and a printed line of length 3.5 cm.
%They should be printed in 9-point Times Roman in single line spacing.}.
%        
%\subsubsection { Third-level heading}
%Please specify references using the conventions
%illustrated below. Each should begin on a new line, and
%second and subsequent lines should be on the same page
%indented by 3 mm.

\subsection*{References}

\begin{description}

\item
Caussinus, H. (1966).
Contribution \`a l'analyse statistique des tableaux de corr\'elation. 
{\it Ann. Fac. Sci. Univ. Toulouse} {\bf 29}, 77\,--\,182.

\item
Kateri, M. and Agresti, A. (2007).
A class of ordinal quasi-symmetry models for square contingency tables.
{\it Statist. Probab. Lett.} {\bf 77}, 598\,--\,603.

\item
Saigusa, Y., Tahata, K. and Tomizawa, S. (2015).
Orthogonal decomposition of symmetry model using the ordinal quasi-symmetry model based on $f$-divergence for square contingency tables.
{\it Statist. Probab. Lett.} {\bf 101}, 33\,--\,37.

\end{description}


\end{document}





