\documentclass[12pt]{article}

\usepackage{microtype}
% \documentstyle{iascars2017}

% \usepackage{iascars2017}

\pagestyle{myheadings} 
\pagenumbering{arabic}
\topmargin 0pt \headheight 23pt \headsep 24.66pt
%\topmargin 11pt \headheight 12pt \headsep 13.66pt
\parindent = 3mm 


\begin{document}


\begin{flushleft}


{\LARGE\bf Computing Entropies with Nested Sampling}


\vspace{1.0cm}

Brendon J. Brewer$^1$

\begin{description}

\item $^1 \;$ Department of Statistics, The University of Auckland,
Private Bag 92019, Auckland 1142, New Zealand

\end{description}

\end{flushleft}

%  ***** ADD ENOUGH VERTICAL SPACE HERE TO ENSURE THAT THE *****
%  ***** ABSTRACT (OR MAIN TEXT) STARTS 5 CM BELOW THE TOP *****

\vspace{0.75cm}

\noindent {\bf Abstract}. The Nested Sampling algorithm, invented in the
mid-2000s by John Skilling, represented a major advance in Bayesian computation.
Whereas Markov Chain Monte Carlo (MCMC) methods are usually effective for sampling
posterior distributions, Nested Sampling also calculates the marginal likelihood
integral used for model comparison, which is a computationally demanding task.
However, there are other kinds of integrals that we might want to compute.
Specifically, the entropy, relative entropy, and mutual information,
which quantify uncertainty and relevance, are all integrals whose form is
inconvenient in most practical applications. I will present my technique,
based on Nested Sampling, for estimating these quantities for probability
distributions that are only accessible via MCMC sampling.
This includes posterior
distributions, marginal distributions, and distributions of derived quantities.
I will present an example from experimental design, where one wants to
optimise the relevance of the data for inference of a parameter.

\vskip 2mm

\noindent {\bf Keywords}.
Bayesian inference, Nested Sampling, Markov Chain Monte Carlo,
Information theory

%\section{ First-level heading}
%The C98 head 1 style leaves a half-line spacing below a
%first-level heading. There should be one blank line above
%a first-level heading.
%        
%\subsection { Second-level heading}
%There should also be one blank line above a second- or
%third-level heading (but no extra space below them).
%
%Do not intent the first paragraph following a heading.
%Second and subsequent paragraphs are indented by one Tab
%character (= 3 mm). If footnotes are used, they should be
%placed at the foot of the page\footnote{ Footnotes are separated
%from the text by a blank line and a printed line of length 3.5 cm.
%They should be printed in 9-point Times Roman in single line spacing.}.
%        
%\subsubsection { Third-level heading}
%Please specify references using the conventions
%illustrated below. Each should begin on a new line, and
%second and subsequent lines should be on the same page
%indented by 3 mm.

\subsection*{References}

\begin{description}

\item Brewer, B. J. (2017).
\textit{Computing Entropies with Nested Sampling.}
Entropy, 19, 422.

\item Skilling, J. (2006).
\textit{Nested Sampling for General Bayesian Computation.}
Bayesian analysis, 1(4), 833-859.

\end{description}

\end{document}





