\documentclass[12pt]{article}
% \documentstyle{iascars2017}

% \usepackage{iascars2017}

\pagestyle{myheadings} 
\pagenumbering{arabic}
\topmargin 0pt \headheight 23pt \headsep 24.66pt
%\topmargin 11pt \headheight 12pt \headsep 13.66pt
\parindent = 3mm 


\begin{document}


\begin{flushleft}


{\LARGE\bf Structure of Members in the Organization to Induce Innovation: Quantitatively Analyze the Capability of the Organization}


\vspace{1.0cm}

Yuji Mizukami$^1$ and Junji Nakano$^2$

\begin{description}

\item $^1 \;$ Nihon University, 1-2-1 Izumicho, Narashino, Chiba 275-8575, Japan

\item $^2 \;$ Institute of Statistical Mathematics, 10-3 Midori-cho, Tachikawa, Tokyo 190-8562, Japan

\end{description}

\end{flushleft}

%  ***** ADD ENOUGH VERTICAL SPACE HERE TO ENSURE THAT THE *****
%  ***** ABSTRACT (OR MAIN TEXT) STARTS 5 CM BELOW THE TOP *****

\vspace{0.75cm}

\noindent {\bf Abstract}. Innovation is the act of creating new value by using "new connection", "new point of view", "new way of thinking", "new usage method" (Schumpeter 1912). In recent years, the promotion of the Innovation has been strongly encouraged. In the field of research, attempts are also being made to create new value through connection between those fields. Moreover, along with the move to promote integration among these research fields, research is being conducted to grasp and promote the degree of them. 
In this research, for the purpose of providing indices for measuring the degree of them, we show indices quantitatively indicating the degree of fusion in different fields and the distance between the fields. Also, we have try to present indices for grasping the whole image based on the random graph.

\vskip 2mm

\noindent {\bf Keywords}.
Research Metrix, Institute Research, Co-author analysis


%\section{ First-level heading}
%The C98 head 1 style leaves a half-line spacing below a
%first-level heading. There should be one blank line above
%a first-level heading.
%        
%\subsection { Second-level heading}
%There should also be one blank line above a second- or
%third-level heading (but no extra space below them).
%
%Do not intent the first paragraph following a heading.
%Second and subsequent paragraphs are indented by one Tab
%character (= 3 mm). If footnotes are used, they should be
%placed at the foot of the page\footnote{ Footnotes are separated
%from the text by a blank line and a printed line of length 3.5 cm.
%They should be printed in 9-point Times Roman in single line spacing.}.
%        
%\subsubsection { Third-level heading}
%Please specify references using the conventions
%illustrated below. Each should begin on a new line, and
%second and subsequent lines should be on the same page
%indented by 3 mm.

\subsection*{References}

\begin{description}

\item
Wagner, C. S., Roessner, J. D., Bobb, K., Klein, J. T., Boyack, K. W., Keyton, J. and Börner, K. (2011).
\textit{Approaches to understanding and measuring interdisciplinary scientific research: A review of the literature, Journal of Informetrics}. Vol. 5, No. 1, pp. 14-26.

\item
Mizukami, Y., Mizutani, Y., Honda, K., Suzuki, S., Nakano, J. (2017).
\textit{An International Research Comparative Study of the Degree of Cooperation between disciplines within mathematics and mathematical sciences, Behaviormetrika},
\textbf{1}, 19 pages, On-line.

\end{description}

\end{document}





