\documentclass[12pt]{article}
% \documentstyle{iascars2017}

% \usepackage{iascars2017}

\pagestyle{myheadings} 
\pagenumbering{arabic}
\topmargin 0pt \headheight 23pt \headsep 24.66pt
%\topmargin 11pt \headheight 12pt \headsep 13.66pt
\parindent = 3mm 


\begin{document}


\begin{flushleft}


{\LARGE\bf Clustering of curves on a spatial domain using a Bayesian partitioning model}


\vspace{1.0cm}

Zhen Zhang$^1$, Chae Young Lim$^2$, Tapabrata Maiti$^3$ and Seiji Kato$^4$

\begin{description}

\item $^1 \;$ Dow AgroSciences. Indianapolis, IN, USA.

\item $^2 \;$ Seoul National University, Seoul, KOREA

\item $^3 \;$ Michigan State University, East Lansing, MI, USA.

\item $^4 \;$ NASA Langley Research Center, Hampton, VA, USA.




\end{description}

\end{flushleft}

%  ***** ADD ENOUGH VERTICAL SPACE HERE TO ENSURE THAT THE *****
%  ***** ABSTRACT (OR MAIN TEXT) STARTS 5 CM BELOW THE TOP *****

\vspace{0.75cm}

\noindent {\bf Abstract}. We propose a Bayesian hierarchical model for spatial clustering of the high-dimensional 
functional data based on the effects of functional covariates. 
We couple the functional mixed-effects model with a generalized spatial partitioning method for: 
(1) identifying subregions for the high-dimensional spatio-functional data; 
(2) improving the computational feasibility via parallel computing over subregions 
or multi-level partitions; and (3) addressing the near-boundary ambiguity in model-based 
spatial clustering techniques. The proposed model extends the existing spatial clustering 
techniques to produce spatially contiguous partitions for spatio-functional data. 
The model successfully captured the regional effects of the atmospheric and 
cloud properties on the spectral radiance measurements. 
This elaborates the importance of considering spatially contiguous partitions 
for identifying regional effects and small-scale variability. 
\vskip 2mm

\noindent {\bf Keywords}.
spatial clustering, Bayesian wavelets, Voronoi tessellation,
 functional covariates


%\section{ First-level heading}
%The C98 head 1 style leaves a half-line spacing below a
%first-level heading. There should be one blank line above
%a first-level heading.
%        
%\subsection { Second-level heading}
%There should also be one blank line above a second- or
%third-level heading (but no extra space below them).
%
%Do not intent the first paragraph following a heading.
%Second and subsequent paragraphs are indented by one Tab
%character (= 3 mm). If footnotes are used, they should be
%placed at the foot of the page\footnote{ Footnotes are separated
%from the text by a blank line and a printed line of length 3.5 cm.
%They should be printed in 9-point Times Roman in single line spacing.}.
%        
%\subsubsection { Third-level heading}
%Please specify references using the conventions
%illustrated below. Each should begin on a new line, and
%second and subsequent lines should be on the same page
%indented by 3 mm.

%\subsection*{References}
%
%\begin{description}
%
%\item
%Barnett, J.A., Payne, R.W. and Yarrow, D. (1990).
%\textit{Yeasts: Characteristics and identification: Second Edition.}
%Cambridge: Cambridge University Press.
%
%\item
%(ed.) Barnett, V., Payne, R. and Steiner, R. (1995).
%\textit{Agricultural Sustainability: Economic, Environmental and
%Statistical Considerations}. Chichester: Wiley.
%
%\item
%Payne, R.W. (1997).
%\textit{Algorithm AS314 Inversion of matrices Statistics},
%\textbf{46}, 295--298.
%
%\item
%Payne, R.W. and Welham, S.J. (1990).
%A comparison of algorithms for combination of information in generally
%balanced designs.
%In: \textit{COMPSTAT90 Proceedings in Computational Statistics}, 297--302.
%Heidelberg: Physica-Verlag.
%
%\end{description}

\end{document}





