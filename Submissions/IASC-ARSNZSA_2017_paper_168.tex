\documentclass[12pt]{article}
% \documentstyle{iascars2017}

% \usepackage{iascars2017}

\pagestyle{myheadings} 
\pagenumbering{arabic}
\topmargin 0pt \headheight 23pt \headsep 24.66pt
%\topmargin 11pt \headheight 12pt \headsep 13.66pt
\parindent = 3mm 


\begin{document}


\begin{flushleft}


{\LARGE\bf Selecting the number of principal components}


\vspace{1.0cm}

Yunjin Choi$^1$ 

\begin{description}

\item $^1 \;$ Department of Statistics and Applied Probability, National University of Singapore, 129801, Singapore


\end{description}

\end{flushleft}

%  ***** ADD ENOUGH VERTICAL SPACE HERE TO ENSURE THAT THE *****
%  ***** ABSTRACT (OR MAIN TEXT) STARTS 5 CM BELOW THE TOP *****

\vspace{0.75cm}

\noindent {\bf Abstract}. Principal Component Analysis (PCA) is one of the most popular methods in multivariate data analysis, which can be applied to covariance matrices. Despite the popularity of the method, there is no widely adopted standard approach to select the number of principal components to retain. To address this issue, we propose a novel method utilizing the hypothesis testing framework and test whether the currently selected principal components capture all the statistically significant signals in the given data set. While existing hypothesis testing approaches do not enjoy the exact type 1 error property and lose power under some scenarios, the proposed method provides an exact type 1 error control along with decent size of power in detecting signals. Central to our work is the post-selection inference framework which facilitates valid inference after data-driven model selection; the proposed hypothesis testing method provides exact type 1 error controls by conditioning on the selection event which leads to the inference. We also introduce a possible extension of the proposed method for high-dimensional data.

\vskip 2mm

\noindent {\bf Keywords}.
Principal component analysis, post-selection inference, hypothesis testing


%\section{ First-level heading}
%The C98 head 1 style leaves a half-line spacing below a
%first-level heading. There should be one blank line above
%a first-level heading.
%        
%\subsection { Second-level heading}
%There should also be one blank line above a second- or
%third-level heading (but no extra space below them).
%
%Do not intent the first paragraph following a heading.
%Second and subsequent paragraphs are indented by one Tab
%character (= 3 mm). If footnotes are used, they should be
%placed at the foot of the page\footnote{ Footnotes are separated
%from the text by a blank line and a printed line of length 3.5 cm.
%They should be printed in 9-point Times Roman in single line spacing.}.
%        
%\subsubsection { Third-level heading}
%Please specify references using the conventions
%illustrated below. Each should begin on a new line, and
%second and subsequent lines should be on the same page
%indented by 3 mm.

\end{document}





