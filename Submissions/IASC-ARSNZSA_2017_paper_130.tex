\documentclass[12pt]{article}
% \documentstyle{iascars2017}

% \usepackage{iascars2017}

\pagestyle{myheadings} 
\pagenumbering{arabic}
\topmargin 0pt \headheight 23pt \headsep 24.66pt
%\topmargin 11pt \headheight 12pt \headsep 13.66pt
\parindent = 3mm 


\begin{document}


\begin{flushleft}


{\LARGE\bf Elastic-Band Transform
	: A New Approach to Multiscale Visualization}


\vspace{1.0cm}

Guebin Choi$^1$ and Hee-Seok Oh$^1$

\begin{description}

\item $^1 \;$ Department of Statistics Seoul National University Seoul 08826, Korea

\end{description}

\end{flushleft}

%  ***** ADD ENOUGH VERTICAL SPACE HERE TO ENSURE THAT THE *****
%  ***** ABSTRACT (OR MAIN TEXT) STARTS 5 CM BELOW THE TOP *****

\vspace{0.75cm}

\noindent {\bf Abstract}. This paper presents a new transformation technique for multiscale visualization of one-dimensional data such as time series and functional data under the concept of the scale-space approach. The proposed method uses a range of regular observations (eye scanning) with varying intervals. The results, termed `elastic-band transform' can be considered as a collection of observations over different intervals of viewing. It is motivated by a way that human looks at an object such as a sequence of data repeatedly in order to overview a global structure of it as well as find some specific features of it.  Some measures based on elastic-bands are discussed for describing characteristics of data, and two-dimensional visualizations induced by the measures are developed for understanding and detecting important structures of data. Furthermore, some statistical applications are studied.   
\vskip 2mm

\noindent {\bf Keywords}.
 Transformation; Visualization; Decomposition; Filter; Time Series;


%\section{ First-level heading}
%The C98 head 1 style leaves a half-line spacing below a
%first-level heading. There should be one blank line above
%a first-level heading.
%        
%\subsection { Second-level heading}
%There should also be one blank line above a second- or
%third-level heading (but no extra space below them).
%
%Do not intent the first paragraph following a heading.
%Second and subsequent paragraphs are indented by one Tab
%character (= 3 mm). If footnotes are used, they should be
%placed at the foot of the page\footnote{ Footnotes are separated
%from the text by a blank line and a printed line of length 3.5 cm.
%They should be printed in 9-point Times Roman in single line spacing.}.
%        
%\subsubsection { Third-level heading}
%Please specify references using the conventions
%illustrated below. Each should begin on a new line, and
%second and subsequent lines should be on the same page
%indented by 3 mm.

\subsection*{References}

\begin{description}
\item Chaudhuri, P. and Marron, J.~S. (1999).  SiZer for exploration of structures in curves.  {\it Journal of the American Statistical Association}, {\bf 94}, 807--823.

\item Donoho, D. L., and Johnstone, I. M. (1994). Ideal spatial adaptation by wavelet shrinkage. {\it Biometrika}, {\bf 81}, 425-455.

\item Dragomiretskiy, K. and Zosso, D. (2014). Variational mode decomposition. {\it IEEE Transactions on Signal Processing}, {\bf 62}, 531--544.

\item Er\"ast\"o, P. and Holmstr\"om, L. (2005). Bayesian multiscale smoothing for making inferences about features in scatter plots. {\it Journal of Computational and Graphical Statistics}, {\bf 14}, 569--589.

\item Fryzlewicz, P. and Oh, H.-S. (2011). Thick pen transformation for time series. {\it Journal of the Royal Statistical Society: Series B (Statistical Methodology)}, {\bf 73}, 499--529.

\item Hannig, J. and Lee, T. C. M. (2006). Robust SiZer for exploration of regression structures and outlier detection. {\it Journal of Computational and Graphical Statistics}, {\bf 15}, 101--117.

\item Hannig, J., Lee, T. and Park, C. (2013). Metrics for SiZer map comparison. {\it Stat}, {\bf 2}, 49--60.

\item Holmstr\"om, L. (2010a). BSiZer.  {\it Wiley Interdisciplinary Reviews: Computational Statistics}, {\bf 2}, 526--534. 

\item Holmstr\"oma, L. (2010b). Scale space methods. {\it Wiley Interdisciplinary Reviews: Computational Statistics}, {\bf 2},150--159.

\item Holmstr\"oma, L. and Pasanena, L. (2017). Statistical scale space methods. {\it International Statistical Review}, {\bf 85}, 1--30.  

\item Huang, N. E., Shen, Z., Long, S. R., Wu, M. C., Shih, H. H., Zheng, Q., ... \& Liu, H. H. (1998). The empirical mode decomposition and the Hilbert spectrum for nonlinear and non-stationary time series analysis. {\it Proceedings of the Royal Society of London A: Mathematical, Physical and Engineering Sciences}, {\bf 454},  903--995. 

\item Lindeberg, T. (1994). {\it Scale-Space Theory in Computer Vision}, Springer Science \& Business Media, New York.


\item Park, C., Hannig, J. and Kang, K. H. (2009). Improved SiZer for time series. {\it Statistica Sinica}, {\bf 19}, 1511--1530. 

\item Park, C, Lee, T. C. and Hannig, J. (2010). Multiscale exploratory analysis of regression quantiles using quantile SiZer.
{\it Journal of Computational and Graphical Statistics}, {\bf 19}, 497--513.

\item Rioul, O. and Vetterli, M. (1991). Wavelets and signal processing. {\it IEEE Signal Processing Magazine},  8(LCAV-ARTICLE-1991-005), 14--38.

\item Vogt, M., \& Dette, H. (2015). Detecting gradual changes in locally stationary processes. The Annals of Statistics, 43(2), 713-740.

\end{description}

\end{document}





