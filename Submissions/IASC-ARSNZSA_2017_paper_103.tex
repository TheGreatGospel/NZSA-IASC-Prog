\documentclass[12pt]{article}
% \documentstyle{iascars2017}

% \usepackage{iascars2017}

\pagestyle{myheadings} 
\pagenumbering{arabic}
\topmargin 0pt \headheight 23pt \headsep 24.66pt
%\topmargin 11pt \headheight 12pt \headsep 13.66pt
\parindent = 3mm 


\begin{document}

\begin{flushleft}


{\LARGE\bf Comparison of Exact and Approximate Testing Procedures in Clinical Trials with Multiple Binary Endpoints}


\vspace{1.0cm}

Takuma Ishihara$^1$ and Kouji Yamamoto$^1$

\begin{description}

\item $^1 \;$ Department of Medical Statistics Osaka City University Graduate School of Medicine, Japan

\end{description}

\end{flushleft}

%  ***** ADD ENOUGH VERTICAL SPACE HERE TO ENSURE THAT THE *****
%  ***** ABSTRACT (OR MAIN TEXT) STARTS 5 CM BELOW THE TOP *****

\vspace{0.75cm}

\noindent {\bf Abstract}. 

%<background> 
In confirmatory clinical trials, the efficacy of a test treatment are sometimes assessed by using multiple primary endpoints.
We consider a trial in which the efficacy of a test treatment is confirmed only when it is superior to control for at least one of the endpoints and not clinically inferior for the remaining endpoints.
Nakazuru et al. (2014) proposed a testing procedure that is applicable to the above case when endpoints are continuous variables.
In this presentation, firstly, we propose a testing procedure in the case that all of the endpoints are binary.

Westfall and Troendle (2008) proposed multivariate permutation tests.
Using this methods, we also propose an exact multiple testing procedure.

Finally, we compare an exact and approximate testing procedures proposed above.
The performance of the proposed procedures was examined through Monte Carlo simulations.

\vskip 2mm

\noindent {\bf Keywords}.
Clinical trial; Multivariate Bernoulli distribution; Non-inferiority; Superiority.
%\section{ First-level heading}
%The C98 head 1 style leaves a half-line spacing below a
%first-level heading. There should be one blank line above
%a first-level heading.
%        
%\subsection { Second-level heading}
%There should also be one blank line above a second- or
%third-level heading (but no extra space below them).
%
%Do not intent the first paragraph following a heading.
%Second and subsequent paragraphs are indented by one Tab
%character (= 3 mm). If footnotes are used, they should be
%placed at the foot of the page\footnote{ Footnotes are separated
%from the text by a blank line and a printed line of length 3.5 cm.
%They should be printed in 9-point Times Roman in single line spacing.}.
%        
%\subsubsection { Third-level heading}
%Please specify references using the conventions
%illustrated below. Each should begin on a new line, and
%second and subsequent lines should be on the same page
%indented by 3 mm.

\subsection*{References}

\begin{description}
\item[] Nakazuru, Y., Sozu, T., Hamada, C. and Yoshimura, I. (2014). A new procedure of one-sided test in clinical trials with multiple endpoints. \textit{Japanese Journal of Biometrics,} \textbf{35}, 17-35.
\item[] Westfall PH and Troendle JF. (2008). Multiple testing with minimal assumptions. \textit{Biometrical Journal,} \textbf{50(5)}, 745-755.
%\item[] Nakazuru, Y., Sozu, T., Hamada, C. and Yoshimura, I. (2014). \textit{Japanese Journal of Biometrics,} \textbf{35}, 17-35.
%\item[] Westfall PH and Troendle JF. (2008). \textit{Biometrical Journal,} \textbf{50(5)}, 745-755.
\end{description}
\end{document}





