\documentclass[12pt]{article}
% \documentstyle{iascars2017}

% \usepackage{iascars2017}

\pagestyle{myheadings} 
\pagenumbering{arabic}
\topmargin 0pt \headheight 23pt \headsep 24.66pt
%\topmargin 11pt \headheight 12pt \headsep 13.66pt
\parindent = 3mm 


\begin{document}


\begin{flushleft}


{\LARGE\bf Author Name Identification for Evaluating Research Performance of Institutes}


\vspace{1.0cm}

Tomokazu Fujino$^1$, Keisuke Honda$^2$ and Hiroka Hamada$^2$

\begin{description}

\item $^1 \;$ Department of Environmental Science, Fukuoka Women's University,
Kasumigaoka, Fukuoka 813-8529, Japan

\item $^2 \;$ Institute of Statistical Mathematics,
Tachikawa, Tokyo 190-8562, Japan

\end{description}

\end{flushleft}

%  ***** ADD ENOUGH VERTICAL SPACE HERE TO ENSURE THAT THE *****
%  ***** ABSTRACT (OR MAIN TEXT) STARTS 5 CM BELOW THE TOP *****

\vspace{0.75cm}

\noindent {\bf Abstract}. In this paper, we propose a new framework to extract a complete list of the articles written by researchers who belong to a specific research or educational institute from an academic document database such as Web of Science and Scopus. In this framework, it is necessary to perform author name identification because the query for the database is based on the author's name to extract documents written before the author comming to the current institute. The framework is based on the latent dirichlet allocation (LDA), which is a kind of topic modeling, and some techniques and indices such as synonym retrieval and inverse document frequency (IDF) are used for enhancing the framework.


\vskip 2mm

\noindent {\bf Keywords}.
Institutional Research, Topic Modeling, Latent Dirichlet Allocation

%\section{ First-level heading}
%The C98 head 1 style leaves a half-line spacing below a
%first-level heading. There should be one blank line above
%a first-level heading.
%        
%\subsection { Second-level heading}
%There should also be one blank line above a second- or
%third-level heading (but no extra space below them).
%
%Do not intent the first paragraph following a heading.
%Second and subsequent paragraphs are indented by one Tab
%character (= 3 mm). If footnotes are used, they should be
%placed at the foot of the page\footnote{ Footnotes are separated
%from the text by a blank line and a printed line of length 3.5 cm.
%They should be printed in 9-point Times Roman in single line spacing.}.
%        
%\subsubsection { Third-level heading}
%Please specify references using the conventions
%illustrated below. Each should begin on a new line, and
%second and subsequent lines should be on the same page
%indented by 3 mm.

\subsection*{References}

\begin{description}

\item Tang, L. and Walsh,J.P. (2010). Bibliometric fingerprints: name disambiguation based on approximate structure equivalence of cognitive maps. \textit{Scientometrics}, 84(3), 763--784.

\item
Strotmann,A., Zhao,D. and Bubela,T. (2009). Author name disambiguation for collaboration network analysis and visualization. \textit{Proc. American Society for Information Science and Technology}, 46(1), 1--20.

\item
Soler,J.M. (2007). Separating the articles of authors with the same name. \textit{Scientometrics}, 72(2), 281--290.

\end{description}

\end{document}





