\documentclass[12pt]{article}
% \documentstyle{iascars2017}

% \usepackage{iascars2017}

\pagestyle{myheadings} 
\pagenumbering{arabic}
\topmargin 0pt \headheight 23pt \headsep 24.66pt
%\topmargin 11pt \headheight 12pt \headsep 13.66pt
\parindent = 3mm 


\begin{document}


\begin{flushleft}


{\LARGE\bf Visual support for imbalanced classification}


\vspace{1.0cm}

Adalbert F.X. Wilhelm$^1$ 

\begin{description}

\item $^1 \;$ Department of Psychology and Methods, Jacobs University, Bremen, Germany

\end{description}

\end{flushleft}

%  ***** ADD ENOUGH VERTICAL SPACE HERE TO ENSURE THAT THE *****
%  ***** ABSTRACT (OR MAIN TEXT) STARTS 5 CM BELOW THE TOP *****

\vspace{0.75cm}

\noindent {\bf Abstract}. Visualisation techniques are widely used to successfully enhance the knowledge discovery in data base processes and a broad spectrum of visualisation approaches for classification problems exist. A core aim of applying visual techniques in the data analysis process is to ease the interaction between algorithms and the user. In general, there are three main stages in the analysis process, namely algorithm selection, model construction, and model evaluation, that can be enhanced by the use of appropriate visualisation techniques.  In particular, context knowledge of the data owner is considered to be of vital importance and often provides a substantial input to improve modelling results. For this purpose, visual displays provide an important means of communication between content expert and data analyst.
Many real life classification problems are characterised by the class imbalance between the various categories of the response variable, e.g. credit risk scoring, fraud detection, concession cases or military conflicts. These imbalances pose a particular challenge to visualisation that go beyond the mere question of image resolution.

In this talk we investigate how visualisation needs and can be adapted to support the further analysis of class-imbalanced data. Looking at continuous and categorical predictors we evaluate the visual comparability of graphical representations for imbalanced data at all three stages of the analysis process. Using the large tool kit of \textbf{R} graphics, we provide a taxonomy of  the effect of class-imbalance on the visual displays for data exploration. Displays of model quality and model comparisons are evaluated with a focus on predictive models derived from ensemble methods.
Here, a specific emphasis is put at the interplay between graphical representations and applying data level methods and algorithmic level methods to handle class imbalances. 
\vskip 2mm

\noindent {\bf Keywords}.
cost-sensitive learning, ensemble methods, over/under-sampling


%\section{ First-level heading}
%The C98 head 1 style leaves a half-line spacing below a
%first-level heading. There should be one blank line above
%a first-level heading.
%        
%\subsection { Second-level heading}
%There should also be one blank line above a second- or
%third-level heading (but no extra space below them).
%
%Do not intent the first paragraph following a heading.
%Second and subsequent paragraphs are indented by one Tab
%character (= 3 mm). If footnotes are used, they should be
%placed at the foot of the page\footnote{ Footnotes are separated
%from the text by a blank line and a printed line of length 3.5 cm.
%They should be printed in 9-point Times Roman in single line spacing.}.
%        
%\subsubsection { Third-level heading}
%Please specify references using the conventions
%illustrated below. Each should begin on a new line, and
%second and subsequent lines should be on the same page
%indented by 3 mm.

%\subsection*{References}
%
%\begin{description}
%
%\item
%Barnett, J.A., Payne, R.W. and Yarrow, D. (1990).
%\textit{Yeasts: Characteristics and identification: Second Edition.}
%Cambridge: Cambridge University Press.
%
%\item
%(ed.) Barnett, V., Payne, R. and Steiner, R. (1995).
%\textit{Agricultural Sustainability: Economic, Environmental and
%Statistical Considerations}. Chichester: Wiley.
%
%\item
%Payne, R.W. (1997).
%\textit{Algorithm AS314 Inversion of matrices Statistics},
%\textbf{46}, 295--298.
%
%\item
%Payne, R.W. and Welham, S.J. (1990).
%A comparison of algorithms for combination of information in generally
%balanced designs.
%In: \textit{COMPSTAT90 Proceedings in Computational Statistics}, 297--302.
%Heidelberg: Physica-Verlag.
%
%\end{description}

\end{document}





