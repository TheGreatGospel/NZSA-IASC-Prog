\documentclass[12pt]{article}
% \documentstyle{iascars2017}

% \usepackage{iascars2017}

\pagestyle{myheadings} 
\pagenumbering{arabic}
\topmargin 0pt \headheight 23pt \headsep 24.66pt
%\topmargin 11pt \headheight 12pt \headsep 13.66pt
\parindent = 3mm 


\begin{document}


\begin{flushleft}


{\LARGE\bf Sparse estimates from dense precision matrix posteriors}


\vspace{1.0cm}

Beatrix Jones and Amir Bashir

\begin{description}

\item  Institute of Natural and Mathematical Sciences, Massey University,
Albany 0745, NZ


\end{description}

\end{flushleft}

%  ***** ADD ENOUGH VERTICAL SPACE HERE TO ENSURE THAT THE *****
%  ***** ABSTRACT (OR MAIN TEXT) STARTS 5 CM BELOW THE TOP *****

\vspace{0.75cm}

\noindent {\bf Abstract}.
A variety of computationally efficient Bayesian models for the covariance matrix of a multivariate Gaussian distribution are available.  However, all produce a relatively dense estimate of the precision matrix, and are therefore unsatisfactory when one wishes to use the precision matrix to consider the conditional independence structure of the data.  This talk considers the posterior of model fit for these covariance models.  We then undertake post-processing of the Bayes point estimate for the precision matrix to produce a sparse model whose  expected fit  lies within the upper 95\% of the posterior over fits.  Extensions to finding sparse differences between inverse covariance matrices are also considered.   We illustrate our findings with moderate dimensional data examples from metabolomics.  



\vskip 2mm

\noindent {\bf Keywords}.
Gaussian graphical models, precision matrices, Bayesian models, metabolomics


%\section{ First-level heading}
%The C98 head 1 style leaves a half-line spacing below a
%first-level heading. There should be one blank line above
%a first-level heading.
%        
%\subsection { Second-level heading}
%There should also be one blank line above a second- or
%third-level heading (but no extra space below them).
%
%Do not intent the first paragraph following a heading.
%Second and subsequent paragraphs are indented by one Tab
%character (= 3 mm). If footnotes are used, they should be
%placed at the foot of the page\footnote{ Footnotes are separated
%from the text by a blank line and a printed line of length 3.5 cm.
%They should be printed in 9-point Times Roman in single line spacing.}.
%        
%\subsubsection { Third-level heading}
%Please specify references using the conventions
%illustrated below. Each should begin on a new line, and
%second and subsequent lines should be on the same page
%indented by 3 mm.

\
\end{document}





