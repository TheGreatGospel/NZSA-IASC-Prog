\documentclass[12pt]{article}
% \documentstyle{iascars2017}

% \usepackage{iascars2017}

\pagestyle{myheadings} 
\pagenumbering{arabic}
\topmargin 0pt \headheight 23pt \headsep 24.66pt
%\topmargin 11pt \headheight 12pt \headsep 13.66pt
\parindent = 3mm 


\begin{document}


\begin{flushleft}


{\LARGE\bf Test for Genomic Imprinting Effects on the X Chromosome}


\vspace{1.0cm}

Wing Kam Fung$^1$ %and Second Author$^2$

\begin{description}

\item $^1 \;$ Department of Statistics and Actuarial Science, The University of Hong Kong, Hong Kong

%\item $^2 \;$ Center for Applied Research in Computer Science,
%Applied Research Laboratory, Anyville, AB 12345, USA

\end{description}

\end{flushleft}

%  ***** ADD ENOUGH VERTICAL SPACE HERE TO ENSURE THAT THE *****
%  ***** ABSTRACT (OR MAIN TEXT) STARTS 5 CM BELOW THE TOP *****

\vspace{0.75cm}

\noindent {\bf Abstract}. Genomic imprinting is an epigenetic phenomenon that the expression of an allele copy depends on its parental origin. This mechanism has been found to play an important role in many diseases. Methods for detecting imprinting effects have been developed primarily for autosomal markers. However, no method is available in the literature to test for imprinting effects on the X chromosome. Therefore, it is necessary to suggest methods for detecting such imprinting effects. In this talk, the parental-asymmetry test on X the chromosome (XPAT) is first developed to test for imprinting for qualitative traits in the presence of association, based on family trios each with both parents and their affected daughter. Then, we propose 1-XPAT to tackle parent-daughter pairs, each with one parent and his/her affected daughter. By simultaneously considering family trios and parent-daughter pairs, C-XPAT is constructed to test for imprinting. Further, we extend the proposed methods to accommodate complete (with both parents) and incomplete (with one parent) nuclear families having multiple daughters of which at least one is affected. Simulations are conducted to assess the performance of the proposed methods under different settings. Simulation results demonstrate that the proposed methods control the size well, irrespective of the inbreeding coefficient in females being zero or nonzero. By incorporating incomplete nuclear families, C-XPAT is more powerful than XPAT using only complete nuclear families. For practical use, these proposed methods are applied to analyze the rheumatoid arthritis data.

\vskip 2mm

\noindent {\bf Keywords}.
Imprinting effects, X chromosome, qualitative traits, nuclear family 
%Keywords for index, separated by commas, without full-stop at end


%\section{ First-level heading}
%The C98 head 1 style leaves a half-line spacing below a
%first-level heading. There should be one blank line above
%a first-level heading.
%        
%\subsection { Second-level heading}
%There should also be one blank line above a second- or
%third-level heading (but no extra space below them).
%
%Do not intent the first paragraph following a heading.
%Second and subsequent paragraphs are indented by one Tab
%character (= 3 mm). If footnotes are used, they should be
%placed at the foot of the page\footnote{ Footnotes are separated
%from the text by a blank line and a printed line of length 3.5 cm.
%They should be printed in 9-point Times Roman in single line spacing.}.
%        
%\subsubsection { Third-level heading}
%Please specify references using the conventions
%illustrated below. Each should begin on a new line, and
%second and subsequent lines should be on the same page
%indented by 3 mm.

%\subsection*{References}

%\begin{description}

%\item
%Barnett, J.A., Payne, R.W. and Yarrow, D. (1990).
%\textit{Yeasts: Characteristics and identification: Second Edition.}
%Cambridge: Cambridge University Press.

%\item
%(ed.) Barnett, V., Payne, R. and Steiner, R. (1995).
%\textit{Agricultural Sustainability: Economic, Environmental and
%Statistical Considerations}. Chichester: Wiley.

%\item
%Payne, R.W. (1997).
%\textit{Algorithm AS314 Inversion of matrices Statistics},
%\textbf{46}, 295--298.

%\item
%Payne, R.W. and Welham, S.J. (1990).
%A comparison of algorithms for combination of information in generally
%balanced designs.
%In: \textit{COMPSTAT90 Proceedings in Computational Statistics}, 297--302.
%Heidelberg: Physica-Verlag.

%\end{description}

\end{document}





