\documentclass[12pt]{article}
% \documentstyle{iascars2017}

% \usepackage{iascars2017}

\pagestyle{myheadings} 
\pagenumbering{arabic}
\topmargin 0pt \headheight 23pt \headsep 24.66pt
%\topmargin 11pt \headheight 12pt \headsep 13.66pt
\parindent = 3mm 


\begin{document}


\begin{flushleft}


{\LARGE\bf Detecting Change-Points in the Stress-Strength Reliability $P(X<Y)$}


\vspace{1.0cm}

Hang Xu$^1$, Philip L.H. Yu$^1$ and Mayer Alvo$^2$

\begin{description}

\item $^1 \;$ Department of Statistics and Actuarial Science, The University of Hong Kong, Hong Kong 

\item $^2 \;$ Department of Mathematics and Statistics, University of Ottawa, Canada
\end{description}

\end{flushleft}

%  ***** ADD ENOUGH VERTICAL SPACE HERE TO ENSURE THAT THE *****
%  ***** ABSTRACT (OR MAIN TEXT) STARTS 5 CM BELOW THE TOP *****

\vspace{0.75cm}

\noindent {\bf Abstract}. We address the statistical problem of detecting change-points in the stress-strength reliability $R=P(X<Y)$ in a sequence of paired variables $(X,Y)$. Without specifying their underlying distributions, we embed this non-parametric problem into a parametric framework and apply the maximum likelihood method via a dynamic programming approach to determine the locations of the change-points in R. Under some mild conditions, we show the consistency and asymptotic properties of the procedure to locate the change-points. Simulation experiments reveal that in comparison with existing parametric and non-parametric change-point detection methods, our proposed method performs well in detecting both single and multiple change-points in R in terms of the accuracy of the location estimation and the computation time. It offers robust and effective detection capability without the need to specify the exact underling distribution of the variables. Applications to real data demonstrate the usefulness of our proposed methodology for detecting the change-points in the stress-strength reliability R. 

\vskip 2mm

\noindent {\bf Keywords}.
 Multiple change-points detection; Stress-strength model; Dynamic programming


%\section{ First-level heading}
%The C98 head 1 style leaves a half-line spacing below a
%first-level heading. There should be one blank line above
%a first-level heading.
%        
%\subsection { Second-level heading}
%There should also be one blank line above a second- or
%third-level heading (but no extra space below them).
%
%Do not intent the first paragraph following a heading.
%Second and subsequent paragraphs are indented by one Tab
%character (= 3 mm). If footnotes are used, they should be
%placed at the foot of the page\footnote{ Footnotes are separated
%from the text by a blank line and a printed line of length 3.5 cm.
%They should be printed in 9-point Times Roman in single line spacing.}.
%        
%\subsubsection { Third-level heading}
%Please specify references using the conventions
%illustrated below. Each should begin on a new line, and
%second and subsequent lines should be on the same page
%indented by 3 mm.

%\subsection*{References}
%
%\begin{description}
%
%\item
%Barnett, J.A., Payne, R.W. and Yarrow, D. (1990).
%\textit{Yeasts: Characteristics and identification: Second Edition.}
%Cambridge: Cambridge University Press.
%
%\item
%(ed.) Barnett, V., Payne, R. and Steiner, R. (1995).
%\textit{Agricultural Sustainability: Economic, Environmental and
%Statistical Considerations}. Chichester: Wiley.
%
%\item
%Payne, R.W. (1997).
%\textit{Algorithm AS314 Inversion of matrices Statistics},
%\textbf{46}, 295--298.
%
%\item
%Payne, R.W. and Welham, S.J. (1990).
%A comparison of algorithms for combination of information in generally
%balanced designs.
%In: \textit{COMPSTAT90 Proceedings in Computational Statistics}, 297--302.
%Heidelberg: Physica-Verlag.
%
%\end{description}

\end{document}





