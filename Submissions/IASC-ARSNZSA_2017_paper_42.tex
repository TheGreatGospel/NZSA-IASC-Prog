\documentclass[12pt]{article}
% \documentstyle{iascars2017}

% \usepackage{iascars2017}

\pagestyle{myheadings} 
\pagenumbering{arabic}
\topmargin 0pt \headheight 23pt \headsep 24.66pt
%\topmargin 11pt \headheight 12pt \headsep 13.66pt
\parindent = 3mm 


\begin{document}

\begin{flushleft}
{\LARGE\bf Comparison of Tests of Mean Difference in Longitudinal Data
Based on Block Resampling Methods}

\vspace{1.0cm}

Hirohito Sakurai$^1$ and Masaaki Taguri$^1$

\begin{description}
\item $^1 \;$ National Center for University Entrance Examinations,
Tokyo 153-8501, Japan
\end{description}
\end{flushleft}

%  ***** ADD ENOUGH VERTICAL SPACE HERE TO ENSURE THAT THE *****
%  ***** ABSTRACT (OR MAIN TEXT) STARTS 5 CM BELOW THE TOP *****

\vspace{0.75cm}

\noindent {\bf Abstract}. 
Let us consider a two-sample problem in longitudinal data, and
discuss comparison of tests of mean difference using
block resampling methods. The testing methods are based on
moving block bootstrap (MBB), circular block bootstrap (CBB)
and stationary bootstrap (SB). These block resampling techniques
are used to approximate the null distributions of the following
four types of test statistics:
sum of absolute values of difference between two mean sequences ($T_1$),
sum of squares of difference between two mean sequences ($T_2$),
area-difference between two mean curves ($T_3$), and
difference of kernel estimators based on two mean sequences ($S_n$).
Our testing algorithm generates blocks of observations in each sample
similar to MBB, CBB or SB, and draws resamples \textit{with replacement}
or \textit{without replacement} from
the mixed blocks which are generated by two samples.
In the context of block resampling, a resample is usually generated
\textit{with replacement}
from blocks of observations,
however our discussion also includes block resampling
\textit{without replacement} similar to permutation analogy
for MBB, CBB and SB, with $T_1$, $T_2$, $T_3$ and $S_n$, respectively.
Monte Carlo simulations are carried out to examine
the empirical level and power of the testing methods.

\vskip 2mm

\noindent {\bf Keywords}.
moving block bootstrap, circular block bootstrap, stationary bootstrap,
with/without replacement, empirical level/power

%\section{ First-level heading}
%The C98 head 1 style leaves a half-line spacing below a
%first-level heading. There should be one blank line above
%a first-level heading.
%        
%\subsection { Second-level heading}
%There should also be one blank line above a second- or
%third-level heading (but no extra space below them).
%
%Do not intent the first paragraph following a heading.
%Second and subsequent paragraphs are indented by one Tab
%character (= 3 mm). If footnotes are used, they should be
%placed at the foot of the page\footnote{ Footnotes are separated
%from the text by a blank line and a printed line of length 3.5 cm.
%They should be printed in 9-point Times Roman in single line spacing.}.
%        
%\subsubsection { Third-level heading}
%Please specify references using the conventions
%illustrated below. Each should begin on a new line, and
%second and subsequent lines should be on the same page
%indented by 3 mm.

\subsection*{References}

\begin{description}

\item
Lahiri, S.~N. (2003).
\textit{Resampling Methods for Dependent Data}.
New York: Springer.

\end{description}

\end{document}





