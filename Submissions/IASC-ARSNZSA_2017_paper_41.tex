\documentclass{article}
\usepackage[utf8]{inputenc}

\title{The Use of Bayesian Networks in Grape Yield Prediction}
\author{Rory Ellis\\[1cm]{\small Supervisors: Daniel Gerhard and Elena Moltchanova}}


\begin{document}

\maketitle

\section{Abstract}
The requirement for predictions to be made earlier in the growing season has become more important, as the opportunity to plan for the wine production and export earlier in the season becomes desirable. The issue with this is there is less information available to those wishing to make early predictions. The analysis in this paper implements a double sigmoidal curve to model the grape growth over the growing season, as this is typically used in agriculture. 

In order to conduct prediction in this study, a Bayesian Network is considered. This allows the opportunity to consider the knowledge of experts in the field, where grape growers would know the growth behaviour of the grapes, as well as using new data to update the Bayesian Network. This information is then implemented in the form of priors, which involves estimating the parameters of the aforementioned double sigmoidal model. Sensitivity Analysis is done in this research, which looks at the impact of prior assumptions (or lack thereof) from experts. Examinations are also made of the value of adding information to the model, as it can be determined whether the precision in the predictions improves as a result of adding data. The results in this analysis are based off simulation studies.
\end{document}
