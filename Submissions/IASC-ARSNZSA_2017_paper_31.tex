\documentclass[12pt]{article}


\pagestyle{myheadings} 
\pagenumbering{arabic}
\topmargin 0pt \headheight 23pt \headsep 24.66pt
%\topmargin 11pt \headheight 12pt \headsep 13.66pt
\parindent = 3mm 

\begin{document}

\begin{flushleft}


\begin{center}
    {\Large\bf Vector Generalized Linear Time Series Models}
\end{center}

\vspace{1.0cm}

Victor Miranda$^1$ and Thomas Yee$^1$

\begin{description}

\item $^1 \;$ Department of Statistics, University of Auckland,
Auckland, NZ.


\end{description}

\end{flushleft}

\vspace{0.75cm}

%%%%%%%%%%%%%%%%%%
\noindent {\bf Abstract}. 
Since the introduction of the ARMA class in the early 1970s many
time series (TS) extensions have been proposed, e.g., vector ARMA
and GARCH-type models for heteroscedasticity. The result has been
a plethora of models having pockets of substructure but little
overriding framework. In this talk we propose a class of TS models
called Vector Generalized Linear Time Series Models (VGLTSM),
which can be thought of as multivariate generalized linear models
directed towards time series data. The crucial VGLM ideas are
constraint matrices, vector responses and covariate-specific
linear predictors, and estimation by iteratively reweighted
least squares and Fisher scoring. The only addition to the VGLM
framework is a log-likelihood that depends on past values. We show
how several popular sub-classes of TS models are accommodated
as special cases of VGLMs, as well as new work that broadens TS
modelling even more. Algorithmic details of its implementation
in {\large \tt{R}}, and properties such as stationarity, parameters
depending on covariates, expected information matrices and
cointegrated TS are surveyed.

\vskip 2mm

\noindent {\bf Keywords}.
VGLM, time series, Fisher scoring.




%%%%%%%%%%%%%%%%%
\subsection*{References}


\begin{description}

\item
Yee, T. W. (2015) 
\textit{Vector Generalized Linear and Additive Models: With an
Implementation in R.} New York, USA: Springer.

\end{description}

\end{document}





