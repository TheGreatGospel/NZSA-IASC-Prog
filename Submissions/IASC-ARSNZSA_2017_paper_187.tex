\documentclass[11pt]{article} % {amsart}

\textwidth=6.5in
\textheight=9in
\oddsidemargin=0.0in
\evensidemargin=0.0in
\topmargin=-0.5in

\begin{document}

\begin{center}
{\Large\bf A Study of the Influence of Articles in the Large-Scale Citation Network}
\end{center}

\begin{center}
Frederick K. H. Phoa and Livia Lin Hsuan Chang\\
{\em Institute of Statistical Science, Academia Sinica, Taipei 115, Taiwan.}\\

\today
\end{center}

\begin{quote} {\bf Abstract:}
Nowadays there are many research metrics at the author-, article-, journal-levels, like the impact factors and many others. However, none of them possess a universally meaningful interpretation on the research influence at all levels, not mentioning that many are subject-biased and consider neighboring relations only. In this work, we introduce a new network-based research metric called the network influence. It utilizes all information in the whole network and it is universal to any levels. Due to its statistical origin, this metric is computationally efficient and statistically interpretable even if one applies it to a large-scale network. This work demonstrates the analysis of networks via network influence using a large-scale citation database called the Web of Science. By just considering the articles among statistics community in 2005-2014, the network influence of all articles are calculated and compared, resulting in a top-ten important articles that are slightly different from the list via impact factors. This metric can be easily extended to author citation network and many similar networks embedded in the Web of Science.
\end{quote}


\end{document}

