\documentclass[12pt]{article}
% \documentstyle{iascars2017}

% \usepackage{iascars2017}

\pagestyle{myheadings} 
\pagenumbering{arabic}
\topmargin 0pt \headheight 23pt \headsep 24.66pt
%\topmargin 11pt \headheight 12pt \headsep 13.66pt
\parindent = 3mm 


\begin{document}


\begin{flushleft}


{\LARGE\bf Bayesian Temporal Density Estimation with Autoregressive Species Sampling Models
	\footnote{This work is a part of the first author's Ph.D. thesis at Seoul National University.
		Research of Seongil Jo was supported by Basic Science Research Program through the National Research Foundation of Korea (NRF) funded by the Ministry of Education (NRF-2017R1D1A3B03035235). Research of Jaeyong Lee was supported by the National Research Foundation of Korea (NRF) grant funded by the Korea government (MEST) (No. 2011-0030811).}}


\vspace{1.0cm}

Youngin Jo$^1$, Seongil Jo$^2$ and Jaeyong Lee$^3$

\begin{description}

\item $^1 \;$ Kakao corporation, Seongnam, 13494, Republic of Korea
\item $^2 \;$ Department of Statistics (Institute of Applied Statistics), Chonbuk National University,
Jeonju, 54896, Republic of Korea 

\item $^3 \;$ Department of Statistics, Seoul National University, Seoul, 08826, Republic of Korea

\end{description}

\end{flushleft}

%  ***** ADD ENOUGH VERTICAL SPACE HERE TO ENSURE THAT THE *****
%  ***** ABSTRACT (OR MAIN TEXT) STARTS 5 CM BELOW THE TOP *****

\vspace{0.75cm}

\noindent {\bf Abstract}. We propose a Bayesian nonparametric (BNP) model, which is built on a class of species sampling models, for estimating density functions of temporal data. In particular, we introduce species sampling mixture models with temporal dependence. To accommodate temporal dependence, we define dependent species sampling models by modeling random support points and weights through an autoregressive model, and then we construct the mixture models based on the collection of these dependent species sampling models. We propose an algorithm to generate posterior samples and present simulation studies to compare the performance of the proposed models with competitors that are based on Dirichlet process mixture models. We apply our method to the estimation of densities for the price of apartment in Seoul, the closing price in Korea Composite Stock Price Index (KOSPI), and climate variables (daily maximum temperature and precipitation) of around the Korean peninsula.

\vskip 2mm

\noindent {\bf Keywords}.
Autoregressive species sampling models; Dependent random probability measures; Mixture models; Temporal structured data


%\section{ First-level heading}
%The C98 head 1 style leaves a half-line spacing below a
%first-level heading. There should be one blank line above
%a first-level heading.
%        
%\subsection { Second-level heading}
%There should also be one blank line above a second- or
%third-level heading (but no extra space below them).
%
%Do not intent the first paragraph following a heading.
%Second and subsequent paragraphs are indented by one Tab
%character (= 3 mm). If footnotes are used, they should be
%placed at the foot of the page\footnote{ Footnotes are separated
%from the text by a blank line and a printed line of length 3.5 cm.
%They should be printed in 9-point Times Roman in single line spacing.}.
%        
%\subsubsection { Third-level heading}
%Please specify references using the conventions
%illustrated below. Each should begin on a new line, and
%second and subsequent lines should be on the same page
%indented by 3 mm.

%\subsection*{References}
%
%\begin{description}
%
%\item
%Barnett, J.A., Payne, R.W. and Yarrow, D. (1990).
%\textit{Yeasts: Characteristics and identification: Second Edition.}
%Cambridge: Cambridge University Press.
%
%\item
%(ed.) Barnett, V., Payne, R. and Steiner, R. (1995).
%\textit{Agricultural Sustainability: Economic, Environmental and
%Statistical Considerations}. Chichester: Wiley.
%
%\item
%Payne, R.W. (1997).
%\textit{Algorithm AS314 Inversion of matrices Statistics},
%\textbf{46}, 295--298.
%
%\item
%Payne, R.W. and Welham, S.J. (1990).
%A comparison of algorithms for combination of information in generally
%balanced designs.
%In: \textit{COMPSTAT90 Proceedings in Computational Statistics}, 297--302.
%Heidelberg: Physica-Verlag.
%
%\end{description}

\end{document}





