\documentclass[11pt]{article}


\begin{document}

\title{R Package for New Two-Stage Methods in Forecasting Time Series with Multiple Seasonality}

\author{
Anupama Lakshmanan \and
Shubhabrata Das\\ Indian Institute of Management Bangalore, \\
Bannerghatta Road, Bangalore 560076, India.  email: \texttt{shubho@iimb.ac.in}
}
\date{\empty}
\maketitle


\begin{center}
\textbf{Abstract}
\end{center}
	
Complex multiple seasonality is an important emerging challenge in time series forecasting. We propose a framework that segregates the task into two stages. In the first stage, the time series is aggregated at the low frequency level (such as daily or weekly) and suitable methods such as regression, ARIMA or TBATS, are used to fit this lower frequency data. In the second stage, additive or multiplicative seasonality at the higher frequency levels may be estimated using classical, or function-based methods. Finally, the estimates from the two stages are combined.

In this work, we build a package for implementing the above two-stage framework for modeling time series with multiple levels of seasonality within R. This would make it convenient to execute and possibly lead to more practitioners and academicians adopting it. The package would allow the user to decide the specific methods to be used in the two stages and also the separation between high and low frequency. Errors are calculated for both model and validation period, which may be selected by the user and model selection choices based on different criterion will be facilitated. Forecast combination may also be integrated with the developed routine. The schematics will be presented along with demonstration of the package in several real data sets.



\vspace{0.1in}
\noindent
Keywords: Additive seasonality, ARIMA, forecast combination, high frequency, low frequency, multiplicative seasonality, polynomial seasonality, regression, TBATS, trigonometric seasonality. 

\end{document}

