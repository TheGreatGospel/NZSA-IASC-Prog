\documentclass[12pt]{article}
% \documentstyle{iascars2017}

% \usepackage{iascars2017}

\pagestyle{myheadings} 
\pagenumbering{arabic}
\topmargin 0pt \headheight 23pt \headsep 24.66pt
%\topmargin 11pt \headheight 12pt \headsep 13.66pt
\parindent = 3mm 


\begin{document}


\begin{flushleft}


{\LARGE\bf An Overview of the Correspondence Analysis Family}


\vspace{1.0cm}

Eric J. Beh$^1$

\begin{description}

\item $^1 \;$ School of Mathematical \& Physical Sciences, University of Newcastle, Callaghan, NSW 2308, Australia

\end{description}

\end{flushleft}

%  ***** ADD ENOUGH VERTICAL SPACE HERE TO ENSURE THAT THE *****
%  ***** ABSTRACT (OR MAIN TEXT) STARTS 5 CM BELOW THE TOP *****

\vspace{0.75cm}

\noindent {\bf Abstract}. 

Correspondence analysis (CA) is well known to be a member of the family of multivariate analysis techniques and is concerned with the visualisation of the association between two or more categorical variables. The classic texts of Greenacre (1984) and Lebart, Morineau and Warwick (1984), for example, provide an excellent technical, practical and historical account of development of CA up to that period. What is less well known is that the literature on CA extends well beyond the traditional approaches that can be found in many multivariate texts and often there are disciplines that redefine the way in which it is performed. For example, the various fields of ecology have successfully germinated variants including {\it canonical correspondence analysis} and {\it detrended correspondence analysis}. However the scope, and literature, of CA is not confined to these examples. Beh and Lombardo (2014, Section 1.6.3) and provide a comprehensive list of members of the ``family'' which, now, number about 50 members. I shall provide an overview of some of the popular, and not-so-popular, members of the CA family.

\vskip 2mm

\noindent {\bf Keywords}.
Correspondence analysis, Multiple CA, Family of analyses


%\section{ First-level heading}
%The C98 head 1 style leaves a half-line spacing below a
%first-level heading. There should be one blank line above
%a first-level heading.
%        
%\subsection { Second-level heading}
%There should also be one blank line above a second- or
%third-level heading (but no extra space below them).
%
%Do not intent the first paragraph following a heading.
%Second and subsequent paragraphs are indented by one Tab
%character (= 3 mm). If footnotes are used, they should be
%placed at the foot of the page\footnote{ Footnotes are separated
%from the text by a blank line and a printed line of length 3.5 cm.
%They should be printed in 9-point Times Roman in single line spacing.}.
%        
%\subsubsection { Third-level heading}
%Please specify references using the conventions
%illustrated below. Each should begin on a new line, and
%second and subsequent lines should be on the same page
%indented by 3 mm.

\subsection*{References}

\begin{description}

\item
Beh, E. J. and Lombardo, R. (2014). {\it Correspondence Analysis: Theory, Practice and New Strategies}. Chichester: Wiley.

\item
Greenacre, M. J. (1984), {\it Theory and Applications of Correspondence Analysis}. London: Academic Press.

\item
Lebart, L., Morineau, A. and Warwick, K. M. (1984). {\it Multivariate Descriptive Statistical Analysis}. New York: John Wiley \& Sons.

\end{description}

\end{document}





