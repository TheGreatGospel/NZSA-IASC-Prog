\documentclass[12pt]{article}
% \documentstyle{iascars2017}

% \usepackage{iascars2017}

\pagestyle{myheadings} 
\pagenumbering{arabic}
\topmargin 0pt \headheight 23pt \headsep 24.66pt
%\topmargin 11pt \headheight 12pt \headsep 13.66pt
\parindent = 3mm 


\begin{document}


\begin{flushleft}


{\LARGE\bf Spatial Surveillance with Scan Statistics by Controlling the False Discovery Rate}


\vspace{1.0cm}

Xun Xiao$^1$

\begin{description}

\item $^1 \;$ Institute of Fundamental Sciences, Massey University,
Palmerston North 4410, New Zealand

\end{description}

\end{flushleft}

%  ***** ADD ENOUGH VERTICAL SPACE HERE TO ENSURE THAT THE *****
%  ***** ABSTRACT (OR MAIN TEXT) STARTS 5 CM BELOW THE TOP *****

\vspace{0.75cm}

\noindent {\bf Abstract}. In this paper, I investigate a false discovery approach based on spatial scan statistics to detect the spatial disease clusters in a geographical region proposed by Li et al. (2016). The incidence of disease is assumed to follow an inhomogeneous Poisson model discussed in Kulldorff (1997). I show that, though spatial scan statistics are highly correlated, the simple Banjamini-Hochberg (linear step-up) procedure can control the false discovery rate of them by proving that the multivariate Poisson distribution satisfies the PRDS condition (positive regression dependence on a subset) in Benjamini and Yekutieli (2001). 

\vskip 2mm

\noindent {\bf Keywords}.
False Discovery Rate, Poisson Distribution, PRDS, Spatial Scan Statistics 

\vskip 10mm

\noindent {\bf References}

\begin{description}

\item
Benjamini, Y. and Yekutieli, D. (2001).
\textit{The control of the false discovery rate in multiple testing under dependency},
Annals of Statistics,
\textbf{29}(4), 1165--1188.


\item
Kulldorff, M. (1997).
\textit{A spatial scan statistic},
Communications in Statistics-Theory and Methods
\textbf{26}(6), 1481--1496.


\item
Li, Y., Shu, L., and Tsung, F. (2016).
\textit{A false discovery approach for scanning spatial disease clusters with arbitrary shapes},
IIE transactions,
\textbf{48}(7), 684--698.

\end{description}


\end{document}





