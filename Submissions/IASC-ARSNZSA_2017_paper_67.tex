\documentclass[12pt]{article}
% \documentstyle{iascars2017}

% \usepackage{iascars2017}

\pagestyle{myheadings} 
\pagenumbering{arabic}
\topmargin 0pt \headheight 23pt \headsep 24.66pt
%\topmargin 11pt \headheight 12pt \headsep 13.66pt
\parindent = 3mm 


\begin{document}


\begin{flushleft}


{\LARGE\bf Geographically Weighted Principal Component Analysis for Spatio-temporal Statistical Dataset}


\vspace{1.0cm}

Narumasa Tsutsumida$^1$, Paul Harris$^2$,  and Alexis Comber$^3$, 

\begin{description}

\item $^1 \;$ Graduate School of Global Environmental Studies, Kyoto University,
Sakyo, Kyoto 6068501, JAPAN

\item $^2 \;$ Sustainable Agricultural Sciences, Rothamsted Research, North Wyke, Okehampton, Devon, EX20 2SB, UK

\item $^3 \;$ School of Geography, University of Leeds, Leeds LS2 9JT, UK


\end{description}

\end{flushleft}

%  ***** ADD ENOUGH VERTICAL SPACE HERE TO ENSURE THAT THE *****
%  ***** ABSTRACT (OR MAIN TEXT) STARTS 5 CM BELOW THE TOP *****

\vspace{0.75cm}

\noindent {\bf Abstract}.  Spatio-temporal statistical datasets are becoming widely available for social, ecomonic, and environmental researches, however it is often difficult to summarize it and undermine hidden spatial/temporal patterns due to its complexity. Geographically weighted principal component analysis (GWPCA), which uses a moving window or kernel and applies localized PCAs over geographical scape, may be worth to do it, while to optimize kernel bandwidth size and to determine the number of component to retain (NCR) were the most concern (Tsutsumida et al (2017)). In this research we determine both of them together simultaneously so as to minimize leave-one-out residual coefficient of variation of GWPCA with changing bandwidth size and NCR.  As a case study we use annual goat population statistics across 341 administrative units in Mongolia in 1990-2012,  and show  spatiotemporal variations in data, especially influenced by natural disasters. 

\vskip 2mm

\noindent {\bf Keywords}.
Geographically weighted model, Spatio-temporal data, Parameter optimization


%\section{ First-level heading}
%The C98 head 1 style leaves a half-line spacing below a
%first-level heading. There should be one blank line above
%a first-level heading.
%        
%\subsection { Second-level heading}
%There should also be one blank line above a second- or
%third-level heading (but no extra space below them).
%
%Do not intent the first paragraph following a heading.
%Second and subsequent paragraphs are indented by one Tab
%character (= 3 mm). If footnotes are used, they should be
%placed at the foot of the page\footnote{ Footnotes are separated
%from the text by a blank line and a printed line of length 3.5 cm.
%They should be printed in 9-point Times Roman in single line spacing.}.
%        
%\subsubsection { Third-level heading}
%Please specify references using the conventions
%illustrated below. Each should begin on a new line, and
%second and subsequent lines should be on the same page
%indented by 3 mm.

\subsection*{References}

\begin{description}

\item
Tsutsumida N., P. Harris, , A. Comber. 2017. The Application of a Geographically Weighted Principal Component Analysis for Exploring Twenty-three Years of Goat Population Change across Mongolia.
\textit{Annals of the American Association of Geographers},
\textbf{107(5)}, 1060--1074.


\end{description}

\end{document}





