\documentclass[12pt]{article}
% \documentstyle{iascars2017}

% \usepackage{iascars2017}

\pagestyle{myheadings} 
\pagenumbering{arabic}
\topmargin 0pt \headheight 23pt \headsep 24.66pt
%\topmargin 11pt \headheight 12pt \headsep 13.66pt
\parindent = 3mm 


\begin{document}


\begin{flushleft}


{\LARGE\bf 
Improvement of Computation for Nonlinear Multivariate Methods}


\vspace{1.0cm}

Masahiro Kuroda$^1$, Yuichi Mori$^1$, and Masaya Iizuka$^2$

\begin{description}
%
\item $^1 \;$ Department of Management, Okayama University of Science,
Okayama, 700-0001, JAPAN
\item $^2 \;$ Admission Center, Okayama University,
Okayama 700-8530, JAPAN
%
\end{description}

\end{flushleft}

%  ***** ADD ENOUGH VERTICAL SPACE HERE TO ENSURE THAT THE *****
%  ***** ABSTRACT (OR MAIN TEXT) STARTS 5 CM BELOW THE TOP *****

\vspace{0.75cm}

\noindent {\bf Abstract}. 
Nonlinear multivariate methods (NL-MM) using optimal scaling as a quantification 
technique can analyze any data including quantitative and qualitative variables. 
The alternating least squares (ALS) algorithm is the most popular iterative 
algorithm in NL-MM. While the algorithm has a stable convergence property, it 
requires many iterations and a large computational cost, especially for a large 
data set involving many qualitative variables, because its convergence is linear. 
It is therefore important to improve the speed of computation when NL-MM with the 
ALS algorithm is applied. Kuroda and his co-workers tried to accelerate the 
convergence of the ALS algorithm using the vector $\varepsilon$ (v$\varepsilon$) 
accelerator. In this talk, the v$\varepsilon$ acceleration for the ALS algorithm 
is implemented in NL-MM, e.g., nonlinear principal component analysis and nonlinear 
factor analysis, and the performances are demonstrated in numerical experiments. 


%If an abstract is included, it should be
%set in the same type as the main text, with the same line width
%and line spacing, starting 15 lines (typewriter) or 5 cm
%(PC) below the top of the print area; otherwise the first
%heading starts here.

\vskip 2mm

\noindent {\bf Keywords}.
Alternating least squares algorithm, Optimal scaling, Acceleration of convergence


%\section{ First-level heading}
%The C98 head 1 style leaves a half-line spacing below a
%first-level heading. There should be one blank line above
%a first-level heading.
%        
%\subsection { Second-level heading}
%There should also be one blank line above a second- or
%third-level heading (but no extra space below them).
%
%Do not intent the first paragraph following a heading.
%Second and subsequent paragraphs are indented by one Tab
%character (= 3 mm). If footnotes are used, they should be
%placed at the foot of the page\footnote{ Footnotes are separated
%from the text by a blank line and a printed line of length 3.5 cm.
%They should be printed in 9-point Times Roman in single line spacing.}.
%        
%\subsubsection { Third-level heading}
%Please specify references using the conventions
%illustrated below. Each should begin on a new line, and
%second and subsequent lines should be on the same page
%indented by 3 mm.

\subsection*{References}

\begin{description}
%
\item
Gifi, A. (1990).
\textit{Nonlinear multivariate analysis}. Wiley. 
%
\item
Kuroda, M., Mori, Y., Iizuka, M. and Sakakihara, M. (2011). 
Acceleration of the alternating least squares algorithm for principal components analysis. 
\textit{Computational Statistics and Data Analysis}, \textbf{55}, 143--153.
%
\item
Mori, Y., Kuroda, M. and Makino, N. (2016).
\textit{Nonlinear principal component analysis and its Applications}.
JSS Research Series in Statistics, Springer.

\end{description}

\end{document}





