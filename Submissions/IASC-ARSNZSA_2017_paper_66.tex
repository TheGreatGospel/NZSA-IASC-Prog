\documentclass[12pt]{article}
% \documentstyle{iascars2017}

% \usepackage{iascars2017}

\pagestyle{myheadings} 
\pagenumbering{arabic}
\topmargin 0pt \headheight 23pt \headsep 24.66pt
%\topmargin 11pt \headheight 12pt \headsep 13.66pt
\parindent = 3mm 

\begin{document}


\begin{flushleft}


{\LARGE\bf Bringing Multimix from Fortran to R}


\vspace{1.0cm}

Murray Jorgensen$^1$ 

\begin{description}

\item $^1 \;$ Department of Mathematical Sciences, Auckland University of Technology,
 55 Wellesley St E, Auckland, 1010, New Zealand
 
\end{description}

\end{flushleft}

\vspace{0.50cm}

\noindent {\bf Abstract}. 
\begin{description}
\item[ Multimix]  is the name for a class of multivariate finite mixture models designed with clustering ({\it unsupervised learning}) in mind. It is also a name for a program to fit these models, written in Fortran77 by Lyn Hunt as part of her Waikato PhD thesis.
\item[ Why convert to R?] 
Although written in the 1990s Multimix is easy to convert to modern GNU Fortran (gfortran) but there are advantages to having an R version available. For users this means a simpler way of reading in the data and describing the form of the model. Also for ongoing development of improvement and modifications of the Multimix models. R's interactive environment provides a more comfortable place for experimentation.
\item[Designing the new program] Rather than attempt any sort of translation of the old code, the new R version of Multimix is designed from the beginning as an R program. In my talk I will describe some of the design decisions made and the reasons for them. A particular concern was that the R version be as fast as possible.
\item[ How to package up the new program?] Two versions of Multimix in R have been developed, a {\em global} version with many global variables employed, and a {\em nested} version restricting the scope of variables to the surrounding function. The pluses and minuses of each approach will be described.
\end{description}
I am conscious that I may not always have made the best design decisions and comments from others will be welcomed.

\vskip 2mm
\noindent {\bf Keywords}.
multivariate finite mixture models, clustering, package, global, local

\end{document}