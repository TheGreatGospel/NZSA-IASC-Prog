\documentclass[12pt]{article}
% \documentstyle{iascars2017}

% \usepackage{iascars2017}

\pagestyle{myheadings} 
\pagenumbering{arabic}
\topmargin 0pt \headheight 23pt \headsep 24.66pt
%\topmargin 11pt \headheight 12pt \headsep 13.66pt
\parindent = 3mm 


\begin{document}


\begin{flushleft}


{\LARGE\bf Relationships between linguistic characteristics and the use of M\={a}ori loanwords in New Zealand English.}


\vspace{1.0cm}

Steven Miller$^1$ and Andreea Calude$^2$

\begin{description}

\item $^1 \;$ Department of Mathematics and Statistics, University of Waikato,
Hamilton, New Zealand

\item $^2 \;$ General and Applied Linguistics Programme, University of Waikato,
Hamilton, New Zealand

\end{description}

\end{flushleft}

%  ***** ADD ENOUGH VERTICAL SPACE HERE TO ENSURE THAT THE *****
%  ***** ABSTRACT (OR MAIN TEXT) STARTS 5 CM BELOW THE TOP *****

\vspace{0.75cm}

\noindent {\bf Abstract}. We present the initial results from a project looking at the linguistic and socio-linguistic characteristics that affect the prevalence of M\={a}ori loanwords in the use of New Zealand English, and describe the paths we see this research taking in the next few years. 

Loanwords are words that originate in one language (the donor language) and enter into, and are productively used within another language (the host language). For our initial research, we were particularly interested in the use of M\={a}ori loanwords in spoken New Zealand English, as found within the Wellington Corpus of Spoken New Zealand English. 

We used generalised linear mixed effects models to determine if there were significant relationships between the linguistic characteristics of the loanwords used / words replaced, demographic features of the speakers, and the ethnicity of the audiences. 

We found that linguistic characteristics of the loanwords and their English counterparts affect the probability of using the loanword for both P\={a}keh\={a} and M\={a}ori speakers, there was a difference in the probability of using a loanword between the sexes for M\={a}ori speakers only, and M\={a}ori speakers moderated the use of loanwords in conversations depending on the ethnicity of their audience.

We will briefly describe the next phase of the research that will use network modelling to characterise the use of M\={a}ori loanwords in written media.

\vskip 2mm

\noindent {\bf Keywords}.
Linguistics, loanwords, M\={a}ori, GLMM
\end{document}





