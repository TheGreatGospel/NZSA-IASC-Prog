\documentclass[12pt]{article}
% \documentstyle{iascars2017}

% \usepackage{iascars2017}

\pagestyle{myheadings} 
\pagenumbering{arabic}
\topmargin 0pt \headheight 23pt \headsep 24.66pt
%\topmargin 11pt \headheight 12pt \headsep 13.66pt
\parindent = 3mm 


\begin{document}


\begin{flushleft}


{\LARGE\bf Canonical Covariance Analysis for Mixed Numerical and Categorical Three-way Three-mode Data}


\vspace{1.0cm}

Jun Tsuchida$^1$ and Hiroshi Yadohisa$^2$

\begin{description}

\item $^1 \;$ Graduate School of Culture and Information Science, Doshisha University,
1-3 Tataramiyakodani, Kyotanave, Kyoto, Japan

\item $^2 \;$ Department of Culture and Information Science, Doshisha University,
1-3 Tataramiyakodani, Kyotanave, Kyoto, Japan

\end{description}

\end{flushleft}

%  ***** ADD ENOUGH VERTICAL SPACE HERE TO ENSURE THAT THE *****
%  ***** ABSTRACT (OR MAIN TEXT) STARTS 5 CM BELOW THE TOP *****

\vspace{0.75cm}

\noindent {\bf Abstract}.
Three-mode three-way data (objects $\times$ variable $\times$ conditions) have been observed in many areas of research. For example, panel data often include values
for the same objects and variables at different times. 
Given two three-mode three-way data
sets, we often investigate two types of factors: common factors, which show the
relationships between the two data sets, and unique factors, which represent
the uniqueness of each data set. In light of this, canonical covariance analysis has been proposed. However, these datasets often have numerical and categorical variables simultaneously. Many multivariate methods for two three-mode thee-way data sets assume that the data has numerical variables only. To overcome this problem,
we propose three-mode three-way canonical covariance analysis with numerical and categorical variables. We use an optimal scaling method (for example, Yong (1987)) for the quantification of categorical data because the values of a categorical variable could not be compared with the value of a numerical variable.

% If an abstract is included, it should be
%set in the same type as the main text, with the same line width
%and line spacing, starting 15 lines (typewriter) or 5 cm
%(PC) below the top of the print area; otherwise the first
%heading starts here.

\vskip 2mm

\noindent {\bf Keywords}.
Alternative least squares, Dimensional reduction, Optimal scaling,Quantification method
%\section{ First-level heading}
%The C98 head 1 style leaves a half-line spacing below a
%first-level heading. There should be one blank line above
%a first-level heading.
%        
%\subsection { Second-level heading}
%There should also be one blank line above a second- or
%third-level heading (but no extra space below them).
%
%Do not intent the first paragraph following a heading.
%Second and subsequent paragraphs are indented by one Tab
%character (= 3 mm). If footnotes are used, they should be
%placed at the foot of the page\footnote{ Footnotes are separated
%from the text by a blank line and a printed line of length 3.5 cm.
%They should be printed in 9-point Times Roman in single line spacing.}.
%        
%\subsubsection { Third-level heading}
%Please specify references using the conventions
%illustrated below. Each should begin on a new line, and
%second and subsequent lines should be on the same page
%indented by 3 mm.

\subsection*{References}

\begin{description}
	
%\item
 %Harshman, R. A. (1970). Foundations of the PARAFAC procedure: Models and
%conditions for an" explanatory" multi-modal factor analysis. \textit{UCLA Working Papers in
%Phonetics}, \textbf{16}, pp. 1--84.
%\item 
%Tucker, L. R. (1966). Some mathematical notes on three-mode factor analysis. \textit{Psychometrika}, \textbf{31}, pp. 279--311.
\item
Young, F. W. (1981). Quantitative analysis of qualitative data. \textit{Psychometrika}, \textbf{46}, pp. 357--388
\end{description}

\end{document}






