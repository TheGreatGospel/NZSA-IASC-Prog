\documentclass{article}
\usepackage[a4paper, total={6.5in, 8.8in}]{geometry}
%\bibliographystyle{abbrv}

\title{Dimensionality Reduction of Multivariate Data for Bayesian analysis}
\author{Anjali Gupta, James M. Curran, Sally Coulson, Christopher M. Triggs}
\date{\today}
\begin{document}
	\maketitle
	In 2004, Aitken and Lucy published an article detailing a two-level likelihood ratio for multivariate trace evidence. This model has been adopted in a number of forensic disciplines such as the interpretation of glass, drugs (MDMA), and ink. Modern instrumentation is capable of measuring many elements in very low quantities and, not surprisingly, forensic scientists wish to exploit the potential of this extra information to increase the weight of this evidence. The issue, from a statistical point of view, is that the increase in the number of variables (dimension) in the problem leads to increased data demand to understand both the variability within a source, and in between sources. Such information will come in time, but usually we don’t have enough. One solution to this problem is to attempt to reduce the dimensionality through methods such as principal component analysis. This practice is quite common in high dimensional machine learning problems. In this talk, I will describe a study where we attempt to quantify the effects of this this approach on the resulting likelihood ratios using data obtained from SEM-EDX instrument.
	
%	\nocite{*}
%	\bibliographystyle{alpha}
%	\bibliography{nzsa2017talk.bib}

\end{document}	