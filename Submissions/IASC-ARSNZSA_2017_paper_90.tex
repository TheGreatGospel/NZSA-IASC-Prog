\documentclass[12pt]{article}
% \documentstyle{iascars2017}

% \usepackage{iascars2017}

\pagestyle{myheadings} 
\pagenumbering{arabic}
\topmargin 0pt \headheight 23pt \headsep 24.66pt
%\topmargin 11pt \headheight 12pt \headsep 13.66pt
\parindent = 3mm 


\begin{document}


\begin{flushleft}


{\LARGE\bf Measure of Departure from Marginal Average Point-Symmetry for Two-Way Contingency Tables with Ordered Categories}


\vspace{1.0cm}

Kiyotaka Iki$^1$

\begin{description}

\item $^1 \;$  Department of Information Sciences, Faculty of Science and Technology, \\
Tokyo University of Science, Japan

\end{description}

\end{flushleft}

%  ***** ADD ENOUGH VERTICAL SPACE HERE TO ENSURE THAT THE *****
%  ***** ABSTRACT (OR MAIN TEXT) STARTS 5 CM BELOW THE TOP *****

\vspace{0.75cm}

\noindent {\bf Abstract}. For the analysis of two-way contingency tables with ordered categories, Yamamoto, Tahata, Suzuki and Tomizawa (2011) considered a measure to represent the degree of departure from marginal point-symmetry. The maximum value of the measure cannot distinguish two kinds of marginal complete asymmetry with respect to the midpoint. The present article proposes a measure which can distinguish two kinds of marginal asymmetry with respect to the midpoint. It also gives large-sample confidence interval for the proposed measure.

\vskip 2mm

\noindent {\bf Keywords}.
Asymmetry, marginal proportional point-symmetry, marginal point-symmetry, measure, model, ordered category

%\section{ First-level heading}
%The C98 head 1 style leaves a half-line spacing below a
%first-level heading. There should be one blank line above
%a first-level heading.
%        
%\subsection { Second-level heading}
%There should also be one blank line above a second- or
%third-level heading (but no extra space below them).
%
%Do not intent the first paragraph following a heading.
%Second and subsequent paragraphs are indented by one Tab
%character (= 3 mm). If footnotes are used, they should be
%placed at the foot of the page\footnote{ Footnotes are separated
%from the text by a blank line and a printed line of length 3.5 cm.
%They should be printed in 9-point Times Roman in single line spacing.}.
%        
%\subsubsection { Third-level heading}
%Please specify references using the conventions
%illustrated below. Each should begin on a new line, and
%second and subsequent lines should be on the same page
%indented by 3 mm.

\subsection*{References}

\begin{description}

\item Tomizawa, S. (1985).
\textit{Biometrical Journal},
{\bf 27}, 895--905.

\item Wall, K.D. and Lienert, G.A. (1976).
\textit{Biometrical Journal},
{\bf 18}, 259--264.

\item Yamamoto, K., Tahata, K., Suzuki, M. and Tomizawa, S. (2011).
\textit{Statistica},
{\bf 71}, 367--380.


\end{description}

\end{document}





