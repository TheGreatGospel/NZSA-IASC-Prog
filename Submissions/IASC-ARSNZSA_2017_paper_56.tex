\documentclass[12pt]{article}
% \documentstyle{iascars2017}

% \usepackage{iascars2017}
\usepackage{amsmath,amsthm,amssymb} 
\pagestyle{myheadings} 
\pagenumbering{arabic}
\topmargin 0pt \headheight 23pt \headsep 24.66pt
%\topmargin 11pt \headheight 12pt \headsep 13.66pt
\parindent = 3mm 
\def\vec#1{\mbox{\boldmath$#1$}}

\begin{document}


\begin{flushleft}


{\LARGE\bf Cluster-wise regression models combined by a quasi-linear function }


\vspace{1.0cm}

Kenichi Hayashi$^{1*}$, Katsuhiro Omae$^{2}$ and Shinto Eguchi$^3$

\begin{description}

\item $^1 \;$ Department of Mathematics, Keio University, Yokohama, Japan

\item $^2 \;$ Department of Statistical Science, The Graduate University for Advanced Studies, Tokyo, Japan

\item $^3 \;$ The Institute of Statistical Mathematics, Tokyo, Japan


\end{description}

\end{flushleft}

%  ***** ADD ENOUGH VERTICAL SPACE HERE TO ENSURE THAT THE *****
%  ***** ABSTRACT (OR MAIN TEXT) STARTS 5 CM BELOW THE TOP *****

\vspace{0.75cm}

\noindent {\bf Abstract}. 
Suppose that there are multiple heterogeneous subgroups in a dataset. 
In the ``Big data'' era, this would be a natural assumption for many fields of application such as medicine, biology, marketing, psychology, etc. 
Then, conventional linear regression models result in not only poor prediction performance but also misleading interpretation of analyses. 
In this study, we propose an extension of cluster-wise regression models  $\phi^{-1}\left(\sum_{k=1}^Kp_k(\vec{x})\phi(\vec{\beta}_k^\top\vec{x})\right)$, where $\phi$ is a strictly increasing function, $\vec{x}\in\mathbb{R}^d$, $\vec{\beta}_k$ is a regression coefficient for $k$th cluster and $p_k(\vec{x})$ is a non-negative function satisfying $\sum_{k=1}^Kp_k(\vec{x})=1$ for any $\vec{x}$. 
We show that the proposed model has flexibility in clustering and ``averaging'' multiple regressors and hence includes the existing methods such as Sp\"{a}th (1981), DeSarbo et al. (1989) as special cases. 
%The algorithm and some numerical examples are given to show how well the proposed method works. 
%If an abstract is included, it should be
%set in the same type as the main text, with the same line width
%and line spacing, starting 15 lines (typewriter) or 5 cm
%(PC) below the top of the print area; otherwise the first
%heading starts here.

\vskip 2mm

\noindent {\bf Keywords}.
Cluster-wise regression, Generalized linear models, Population heterogeneity


%\section{ First-level heading}
%The C98 head 1 style leaves a half-line spacing below a
%first-level heading. There should be one blank line above
%a first-level heading.
%        
%\subsection { Second-level heading}
%There should also be one blank line above a second- or
%third-level heading (but no extra space below them).
%
%Do not intent the first paragraph following a heading.
%Second and subsequent paragraphs are indented by one Tab
%character (= 3 mm). If footnotes are used, they should be
%placed at the foot of the page\footnote{ Footnotes are separated
%from the text by a blank line and a printed line of length 3.5 cm.
%They should be printed in 9-point Times Roman in single line spacing.}.
%        
%\subsubsection { Third-level heading}
%Please specify references using the conventions
%illustrated below. Each should begin on a new line, and
%second and subsequent lines should be on the same page
%indented by 3 mm.

\subsection*{References}

\begin{description}

\item
DeSarbo, W.S., Oliver, R.L., and Rangaswamy, A. (1989). 
A simulated annealing methodology for clusterwise linear regression. 
\textit{Psychometrika},
\textbf{54}, 707--736.

\item
Sp\"{a}th, H. (1979). 
Algorithm 39: Clusterwise linear regression. 
\textit{Computing},
\textbf{22}, 367--373.

\end{description}

\end{document}





