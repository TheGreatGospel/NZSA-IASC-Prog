\documentclass[12pt]{article}
% \documentstyle{iascars2017}

% \usepackage{iascars2017}

\pagestyle{myheadings}
\pagenumbering{arabic}
\topmargin 0pt \headheight 23pt \headsep 24.66pt
%\topmargin 11pt \headheight 12pt \headsep 13.66pt
\parindent = 3mm


\begin{document}
\begin{flushleft}
{\LARGE\bf LSMM: A statistical approach to integrating functional annotations with genome-wide association studies}
\vspace{1.0cm}

Jingsi Ming$^1$, Mingwei Dai$^{2,5}$, Mingxuan Cai$^1$, Xiang Wan$^3$, Jin Liu$^4$ and Can Yang$^5$ 
\begin{description}
\item $^1 \;$ Department of Mathematics, Hong Kong Baptist University, Hong Kong
\item $^2 \;$ School of Mathematics and Statistics, Xi’an Jiaotong University, Xi’an, China
\item $^3 \;$ Department of Computer Science, Hong Kong Baptist University, Hong Kong
\item $^4 \;$ Centre for Quantitative Medicine, Duke-NUS Medical School, Singapore
\item $^5 \;$ Department of Mathematics, The Hong Kong University of Science and Technology, Hong Kong
\end{description}
\end{flushleft}
%  ***** ADD ENOUGH VERTICAL SPACE HERE TO ENSURE THAT THE *****
%  ***** ABSTRACT (OR MAIN TEXT) STARTS 5 CM BELOW THE TOP *****
\vspace{0.75cm}
\noindent {\bf Abstract}. Thousands of risk variants underlying complex phenotypes have been identified in genome-wide association studies (GWAS). However, there are two major challenges towards fully characterizing the biological basis of complex diseases. First, many complex traits are suggested to be highly polygenic, whereas a large proportion of risk variants with small effects remains unknown. Second, the functional roles of the majority of GWAS hits in the non-coding region is largely unclear. In this paper, we propose a latent sparse mixed model (LSMM) to address the challenges by integrating functional annotations with summary statistics from GWAS. An efficient variational expectation-maximization (EM) algorithm is developed. We conducted comprehensive simulation studies and then applied it to 30 GWAS of complex phenotypes integrating 9 genic annotation categories and 127 tissue-specific functional annotations from the Roadmap project. The results demonstrate that LSMM is not only able to increase the statistical power to identify risk variants, but also provide a deeper understanding of genetic architecture of complex traits by detecting relevant functional annotations.
\vskip 2mm
\noindent {\bf Keywords}.
GWAS, functional annotations, variational inference


%\section{ First-level heading}
%The C98 head 1 style leaves a half-line spacing below a
%first-level heading. There should be one blank line above
%a first-level heading.
%
%\subsection { Second-level heading}
%There should also be one blank line above a second- or
%third-level heading (but no extra space below them).
%
%Do not intent the first paragraph following a heading.
%Second and subsequent paragraphs are indented by one Tab
%character (= 3 mm). If footnotes are used, they should be
%placed at the foot of the page\footnote{ Footnotes are separated
%from the text by a blank line and a printed line of length 3.5 cm.
%They should be printed in 9-point Times Roman in single line spacing.}.
%
%\subsubsection { Third-level heading}
%Please specify references using the conventions
%illustrated below. Each should begin on a new line, and
%second and subsequent lines should be on the same page
%indented by 3 mm.



\end{document}





