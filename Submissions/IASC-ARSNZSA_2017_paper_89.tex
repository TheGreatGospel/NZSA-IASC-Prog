\documentclass[12pt]{article}
% \documentstyle{iascars2017}

% \usepackage{iascars2017}

\pagestyle{myheadings}
\pagenumbering{arabic}
\topmargin 0pt \headheight 23pt \headsep 24.66pt
%\topmargin 11pt \headheight 12pt \headsep 13.66pt
\parindent = 3mm


\begin{document}


\begin{flushleft}


{\LARGE\bf Interactive visualization of aggregated symbolic data}


\vspace{1.0cm}

Yoshikazu YAMAMOTO$^1$ , Junji NAKANO$^2$ and Nobuo SHIMIZU$^2$

\begin{description}

\item $^1 \;$
Tokushima Bunri University,
1314-1 Shido, Sanuki-city, Kagawa 769-2193, Japan

\item $^2 \;$
The Institute of Statistical Mathematics,
10-3 Midori-cho, Tachikawa, Tokyo 190-8562, Japan

\end{description}

\end{flushleft}

%  ***** ADD ENOUGH VERTICAL SPACE HERE TO ENSURE THAT THE *****
%  ***** ABSTRACT (OR MAIN TEXT) STARTS 5 CM BELOW THE TOP *****

\vspace{0.75cm}

\noindent {\bf Abstract}.
When we have new ``big data'', the first step may be to visualize them.
For visualizing continuous multivariate data, interactive parallel coordinate plot is known to be appropriate.
However, the number of data is huge and some variables are categorical, a simple parallel coordinate plot is not available.
We propose to divide big data into rather small groups and summarize them as aggregated symbolic data (ASD), and visualize them by triangular arranged parallel coordinate plots.

We have developed a statistical graphics software for this purpose.
Our software equips interactive operations such as selection and linked highlighting, and is written by Java, R, and big data processing technologies such as Apache Hadoop and Apache Spark.

Aggregated symbolic data is a set of descriptive statistics calculated by up to second order moments of variables in each group.
We also propose further summarization of ASD to describe characteristics of each variable and a pair of variables for visualizing the difference among ASDs.
Real example data are visualized by our software and interpreted intuitively.

\vskip 2mm

\noindent {\bf Keywords}.
Apache Hadoop, Apache Spark, Parallel coordinate plot, Symbolic data analysis.


%\section{ First-level heading}
%The C98 head 1 style leaves a half-line spacing below a
%first-level heading. There should be one blank line above
%a first-level heading.
%
%\subsection { Second-level heading}
%There should also be one blank line above a second- or
%third-level heading (but no extra space below them).
%
%Do not intent the first paragraph following a heading.
%Second and subsequent paragraphs are indented by one Tab
%character (= 3 mm). If footnotes are used, they should be
%placed at the foot of the page\footnote{ Footnotes are separated
%from the text by a blank line and a printed line of length 3.5 cm.
%They should be printed in 9-point Times Roman in single line spacing.}.
%
%\subsubsection { Third-level heading}
%Please specify references using the conventions
%illustrated below. Each should begin on a new line, and
%second and subsequent lines should be on the same page
%indented by 3 mm.


\end{document}
