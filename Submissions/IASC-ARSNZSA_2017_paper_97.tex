\documentclass[12pt]{article}
% \documentstyle{iascars2017}

% \usepackage{iascars2017}

\pagestyle{myheadings} 
\pagenumbering{arabic}
\topmargin 0pt \headheight 23pt \headsep 24.66pt
%\topmargin 11pt \headheight 12pt \headsep 13.66pt
\parindent = 3mm 


\begin{document}


\begin{flushleft}


{\LARGE\bf Analysis of Spatial Data with a 
Gaussian Mixture Markov Random Field Model
}  
%{\LARGE\bf Example of a ISI-IASC-ARS/NZSA Paper
%with Title in Large Boldface Type}


\vspace{1.0cm}

Wataru Sakamoto
%First Author$^1$ and Second Author$^2$

\begin{description}

\item Graduate School of Environmental and Life Science, Okayama University, 700-8530, Japan
%\item $^1 \;$ Department of Computer Science, State University,
%Anyville, AB 12345, USA

%\item $^2 \;$ Center for Applied Research in Computer Science,
%Applied Research Laboratory, Anyville, AB 12345, USA

\end{description}

\end{flushleft}

%  ***** ADD ENOUGH VERTICAL SPACE HERE TO ENSURE THAT THE *****
%  ***** ABSTRACT (OR MAIN TEXT) STARTS 5 CM BELOW THE TOP *****

\vspace{0.75cm}

\noindent {\bf Abstract}. 
In spatial data, detecting regions with higher relative risk is of primary interest.  
%Fitting mixture models would be one of useful methods to detect locations with higher risk.
%The mixture weight for each location indicate directly the probability 
%that it would belongs to a component with greater mean.
%However, it would be hard to detect aggregated regions with higher risk 
%and give meaningful interpretation for spatial structure 
%by fitting independent mixture models. 
A latent Markov random field model with Gaussian mixture component is introduced,
in which the probit or the logit of the mixture weight for each location follows 
a Gaussian Markov random field such as an intrinsic auto-regressive model (Besag \textit{et al.}, 1991). 
A mixture model with spatially correlated weights was proposed by Fern\'andez and Green (2002), 
and our modeling with Gaussian mixture Markov random field 
can be extended to the cases of involving covariate and random effects. 
Parameters are estimated by a Bayesian approach, 
and the posterior mean of the mixture weight for each location, which varies smoothly, 
gives meaningful interpretation for spatial structure.  
Our computation was conducted with R Stan package, 
in which the Hamiltonian Monte Carlo method is implemented.
Some applications to disease mapping data are illustrated.

%If an abstract is included, it should be
%set in the same type as the main text, with the same line width
%and line spacing, starting 15 lines (typewriter) or 5 cm
%5(PC) below the top of the print area; otherwise the first
%heading starts here.

\vskip 2mm

\noindent {\bf Keywords}.
Bayesian modeling, spatial cluster detection, spatial correlation
%Keywords for index, separated by commas, without full-stop at end


%\section{ First-level heading}
%The C98 head 1 style leaves a half-line spacing below a
%first-level heading. There should be one blank line above
%a first-level heading.
%        
%\subsection { Second-level heading}
%There should also be one blank line above a second- or
%third-level heading (but no extra space below them).
%
%Do not intent the first paragraph following a heading.
%Second and subsequent paragraphs are indented by one Tab
%character (= 3 mm). If footnotes are used, they should be
%placed at the foot of the page\footnote{ Footnotes are separated
%from the text by a blank line and a printed line of length 3.5 cm.
%They should be printed in 9-point Times Roman in single line spacing.}.
%        
%\subsubsection { Third-level heading}
%Please specify references using the conventions
%illustrated below. Each should begin on a new line, and
%second and subsequent lines should be on the same page
%indented by 3 mm.

\subsection*{References}

\begin{description}

\item 
Fern\'andez, C. and Green, P. J. (2002). 
Modelling spatially correlated data via mixtures: a Bayesian approach. 
\textit{J. Roy. Statist. Soc. B}, \textbf{64}, 805--826. 

\item 
Besag, J., York, J. and Molli\'e, A. (1991). 
Bayesian image restoration, with two applications in spatial statistics.
\textit{Ann. Inst. Statist. Math.}, \textbf{43}, 1--59. 

\item 
Rue, H. and Held, L. (2005)
\textit{Gaussian Markov Random Fields: Theory and Applications.}
Chapman and Hall.

\end{description}

\end{document}





