\documentclass[12pt]{article}
% \documentstyle{iascars2017}

% \usepackage{iascars2017}
%%%%%%%%%%%%%%%%%%%%%%%%%%%%%%%%%
%\usepackage{amsmath,amsthm,amssymb}
%\theoremstyle{definition}
%\newtheorem*{defn*}{Definition}
%%%%%%%%%%%%%%%%%%%%%%%%%%%%%%%%%

\pagestyle{myheadings} 
\pagenumbering{arabic}
\topmargin 0pt \headheight 23pt \headsep 24.66pt
%\topmargin 11pt \headheight 12pt \headsep 13.66pt
\parindent = 3mm 


\begin{document}


\begin{flushleft}


%{\LARGE\bf Example of a ISI-IASC-ARS/NZSA Paper with Title in Large Boldface Type}
{\LARGE\bf How does the textile set describe geometric structures of data?}

\vspace{1.0cm}

%First Author$^1$ and Second Author$^2$
Ushio Tanaka$^1$ and Tomonari Sei$^2$

\begin{description}

%\item $^1 \;$ Department of Computer Science, State University, Anyville, AB 12345, USA
\item $^1 \;$ Department of Mathematics and Information Sciences, Osaka Prefecture University, Osaka, Japan 

%\item $^2 \;$ Center for Applied Research in Computer Science, Applied Research Laboratory, Anyville, AB 12345, USA
\item $^2 \;$ Department of Mathematical Informatics, The University of Tokyo, Tokyo, Japan 

\end{description}

\end{flushleft}

%  ***** ADD ENOUGH VERTICAL SPACE HERE TO ENSURE THAT THE *****
%  ***** ABSTRACT (OR MAIN TEXT) STARTS 5 CM BELOW THE TOP *****

\vspace{0.75cm}

\noindent {\bf Abstract}. 
%If an abstract is included, it should be set in the same type as the main text, with the same line width and line spacing, starting 15 lines (typewriter) or 5 cm (PC) below the top of the print area; otherwise the first heading starts here.
The textile set is defined from the textile plot proposed by Kumasaka and Shibata (2007, 2008), which is a powerful tool for visualizing high dimensional data. The textile plot is based on a parallel coordinate plot, where the ordering, locations and scales of each axis are simultaneously chosen so that all connecting lines, each of which signifies an observation, are aligned as horizontally as possible. The textile plot transforms a data matrix in order to delineate a parallel coordinate plot. 
%The aim of the present study is to investigate the structure of the textile set from a geometric point of view. 
Using the geometric properties of the textile set derived by Sei and Tanaka (2015), we show that the textile set describes an intrinsically geometric structures of data.    

\vskip 2mm

\noindent {\bf Keywords}.
%Keywords for index, separated by commas, without full-stop at end
%Data matrix, Parallel coordinate plot, Textile plot, Textile set, Differentiable manifold, Analytic geometry  
Parallel coordinate plot, Textile set, Differentiable manifold  

%\section{ First-level heading}
%The C98 head 1 style leaves a half-line spacing below a first-level heading. There should be one blank line above a first-level heading.
%%%%%%%%%%%%%%%%%%
%\section*{Textile set}
%\begin{defn*}
%\end{defn*}
%%%%%%%%%%%%%%%%%%
%\subsection { Second-level heading}
%There should also be one blank line above a second- or third-level heading (but no extra space below them).
%
%Do not intent the first paragraph following a heading. Second and subsequent paragraphs are indented by one Tab character (= 3 mm). If footnotes are used, they should be placed at the foot of the page\footnote{ Footnotes are separated from the text by a blank line and a printed line of length 3.5 cm. They should be printed in 9-point Times Roman in single line spacing.}.
%        
%\subsubsection { Third-level heading}
%Please specify references using the conventions illustrated below. Each should begin on a new line, and second and subsequent lines should be on the same page indented by 3 mm.

\subsection*{References}

\begin{description}

%\item
%Barnett, J.A., Payne, R.W. and Yarrow, D. (1990). \textit{Yeasts: Characteristics and identification: Second Edition.} Cambridge: Cambridge University Press.
%
%\item
%(ed.) Barnett, V., Payne, R. and Steiner, R. (1995). \textit{Agricultural Sustainability: Economic, Environmental and Statistical Considerations}. Chichester: Wiley.
%
%\item
%Payne, R.W. (1997). \textit{Algorithm AS314 Inversion of matrices Statistics}, \textbf{46}, 295--298.
%
%\item
%Payne, R.W. and Welham, S.J. (1990). A comparison of algorithms for combination of information in generally balanced designs. In: \textit{COMPSTAT90 Proceedings in Computational Statistics}, 297--302. Heidelberg: Physica-Verlag.
%
\item
Kumasaka, N. and Shibata, R. (2007). The Textile Plot Environment, \textit{Proceedings of the Institute of Statistical Mathematics}, \textbf{55}, 47--68.
\item
Kumasaka, N. and Shibata, R. (2008). High-dimensional data visualisation: The textile plot, \textit{Computational Statistics and Data Analysis}, \textbf{52}, 3616--3644.
\item
Sei, T. and Tanaka, U. (2015). Geometric Properties of Textile Plot: \textit{Geometric Science of Information}, \textit{Lecture Notes in Computer Science}, \textbf{9389}, 732--739.

\end{description}

\end{document}





