\documentclass[12pt]{article}
% \documentstyle{iascars2017}

% \usepackage{iascars2017}

\pagestyle{myheadings} 
\pagenumbering{arabic}
\topmargin 0pt \headheight 23pt \headsep 24.66pt
%\topmargin 11pt \headheight 12pt \headsep 13.66pt
\parindent = 3mm 


\begin{document}


\begin{flushleft}


{\LARGE\bf Statistical Modelling and Analysis of Cosmic Microwave Background Data}


\vspace{1.0cm}

Andriy Olenko 

\begin{description}

\item Department of Mathematics and Statistics, La Trobe University, Bundoora, Victoria 3086, Australia

\end{description}

\end{flushleft}

%  ***** ADD ENOUGH VERTICAL SPACE HERE TO ENSURE THAT THE *****
%  ***** ABSTRACT (OR MAIN TEXT) STARTS 5 CM BELOW THE TOP *****

\vspace{0.75cm}


\noindent {\bf Keywords}.
random fields, spatial statistics, cosmic microwave background data, R package

\vspace{1.0cm}

Analysis of the Cosmic Microwave Background
(CMB) radiation is a remarkable research area in cosmology whose results won
two Nobel prizes in physics in 1978 and 2006 for the discovery of the CMB
radiation and its anisotropy. Spurred on by a wealth of satellite data,
intensive investigations in the past few years have resulted in many
relevant physical and mathematical formalisms to describe and characterise
CMB radiation. At the same time, these investigations have raised a number
of challenges, theoretical and practical.  
Studies of deviations from isotropy and Gaussianity, the two
	fundamental assumptions of cosmological models of the early Universe, form
	the core of recent experimental and theoretical research in cosmology.

Recent results on modelling CMB evolution and approximation of corresponding random fields will be discussed. Some new approaches to test Gaussianity using multifractality will be illustrated using CMB data. Finally, a new R package for CMB data will be presented.
  
The presentation is based on joint research with Vo Anh (QUT), N.Leonenko (Cardiff university), P.Broadbridge, D. Fryer, Yu.G. Wang (La Trobe University). This research was supported under the Australian Research Council's Discovery Project DP160101366.

\subsection*{References}

\begin{description}

\item
Anh, Vo,  Broadbridge, P., Olenko, A., Wang Yu.G. On approximation for fractional stochastic partial differential equations on the sphere. Submitted.

\end{description}

\end{document}





