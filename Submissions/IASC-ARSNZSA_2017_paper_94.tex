\documentclass[12pt]{article}
% \documentstyle{iascars2017}

% \usepackage{iascars2017}

\pagestyle{myheadings} 
\pagenumbering{arabic}
\topmargin 0pt \headheight 23pt \headsep 24.66pt
%\topmargin 11pt \headheight 12pt \headsep 13.66pt
\parindent = 3mm 


\begin{document}


\begin{flushleft}


 {\LARGE\bf Estimating causal structures for continuous and discrete variables
}


\vspace{1.0cm}

Mako Yamayoshi$^1$ and Hiroshi Yadohisa$^2$

\begin{description}

\item $^1 \;$ Graduate school of Culture and Information Science, Doshisha University, Kyotanabe, Kyoto, JAPAN
\item $^2 \;$ Faculty of Culture and Information Science, Doshisha University, Kyotanabe, Kyoto, JAPAN

\end{description}

\end{flushleft}

%  ***** ADD ENOUGH VERTICAL SPACE HERE TO ENSURE THAT THE *****
%  ***** ABSTRACT (OR MAIN TEXT) STARTS 5 CM BELOW THE TOP *****

\vspace{0.75cm}

\noindent {\bf Abstract}.

Structural equation models have been used extensively for continuous variable data to find causal structures. In such a framework, the Linear Non- Gaussian Acyclic Model (LiNGAM) could enable finding a whole causal model (Shimizu et al., 2006). However, in many desciplines, the data include both continuous and discrete variables. LiNGAM could fail to capture the actual causal relationship for such data because it handles both discrete and continuous variables as continuous. Therefore, it is necessary to improve the estimation method for causal structures in such conditions. 

 In this study, we propose a method to find causal structures for continuous and discrete variables. To overcome the problems of the existing method, we use the Link function. Using simulation studies, we show that the proposed method performs more efficiently for data that includes continuous and discrete variables. 


  

\vskip 2mm
\noindent {\bf Keywords}. Causal direction, Latent variables, Link function, SEM, LiNGAM



%\section{ First-level heading}
%The C98 head 1 style leaves a half-line spacing below a
%first-level heading. There should be one blank line above
%a first-level heading.
%        
%\subsection { Second-level heading}
%There should also be one blank line above a second- or
%third-level heading (but no extra space below them).
%
%Do not intent the first paragraph following a heading.
%Second and subsequent paragraphs are indented by one Tab
%character (= 3 mm). If footnotes are used, they should be
%placed at the foot of the page\footnote{ Footnotes are separated
%from the text by a blank line and a printed line of length 3.5 cm.
%They should be printed in 9-point Times Roman in single line spacing.}.
%        
%\subsubsection { Third-level heading}
%Please specify references using the conventions
%illustrated below. Each should begin on a new line, and
%second and subsequent lines should be on the same page
%indented by 3 mm.

\subsection*{References}


\begin{description}

\iffalse
\item
Barnett, J.A., Payne, R.W. and Yarrow, D. (1990).
\textit{Yeasts: Characteristics and identification: Second Edition.}
Cambridge: Cambridge University Press.
\fi

\item[] S. Shimizu, P.O. Hoyer, A. Hyv\"{a}rinen, and A. Kerminen (2006). A linear non-Gaussian acyclic model for causal discovery. \textit{The Journal of \\Machine Learning Research}, vol. 7, pp. 2003-2030.

%\item[] G. Park, G. Raskutti (2015). Learning Large-Scale DAG Models based on OverDispersion Scoring. \textit{The Journal of \\Machine Learning Research}, 

		   
		   \iffalse
\item
(ed.) Barnett, V., Payne, R. and Steiner, R. (1995).
\textit{Agricultural Sustainability: Economic, Environmental and
Statistical Considerations}. Chichester: Wiley.

\item
Payne, R.W. (1997).
\textit{Algorithm AS314 Inversion of matrices Statistics},
\textbf{46}, 295--298.

\item
Payne, R.W. and Welham, S.J. (1990).
A comparison of algorithms for combination of information in generally
balanced designs.
In: \textit{COMPSTAT90 Proceedings in Computational Statistics}, 297--302.
Heidelberg: Physica-Verlag.
\fi
\end{description}

\end{document}





