\documentclass[12pt]{article}
% \documentstyle{iascars2017}

% \usepackage{iascars2017}

\pagestyle{myheadings} 
\pagenumbering{arabic}
\topmargin 0pt \headheight 23pt \headsep 24.66pt
%\topmargin 11pt \headheight 12pt \headsep 13.66pt
\parindent = 3mm 


\begin{document}


\begin{flushleft}


{\LARGE\bf Bayesian Analyses of Non-homogeneous Gaussian Hidden Markov Models}


\vspace{1.0cm}

Shin Sato$^1$ and Darfiana Nur$^1$

\begin{description}

\item $^1 \;$ College of Science and Engineering, Flinders University at Tonsley, 1284 South Rd, Clovelly Park, SA 5042, Australia

\end{description}

\end{flushleft}

%  ***** ADD ENOUGH VERTICAL SPACE HERE TO ENSURE THAT THE *****
%  ***** ABSTRACT (OR MAIN TEXT) STARTS 5 CM BELOW THE TOP *****

\vspace{0.75cm}

\noindent {\bf Abstract}. We investigate a non-homogeneous Gaussian hidden Markov model where the model assumes the transition probabilities between the hidden states depend on each discrete-time. The methodology of the statistical inference for the model follows the Bayesian approach implementing the Markov chain Monte Carlo (MCMC) methods for parameter estimation. The methods include: the Metropolis-Hastings, the delayed rejection Metropolis-Hastings, the multiple-try Metropolis-Hastings, and the adaptive Metropolis algorithms.

For simulation studies, we have successfully implemented all the algorithms proposed on the simulated data set that was investigated by Diebold et al. (1994), although we had been faced with the difficulties of estimating each parameter due to the large noises in the data. For a case study, the model was implemented on a data set of the monthly US 3-month treasury bill rates with six financial exogenous variables in which the settings are identical to that of Meligkotsidou and Dellaportas's (2011), except for the algorithm.

\vskip 2mm

\noindent {\bf Keywords}.
Non-homogeneous hidden Markov model, Bayesian inference, Markov chain Monte Carlo methods, Metropolis-Hastings algorithms


%\section{ First-level heading}
%The C98 head 1 style leaves a half-line spacing below a
%first-level heading. There should be one blank line above
%a first-level heading.
%        
%\subsection { Second-level heading}
%There should also be one blank line above a second- or
%third-level heading (but no extra space below them).
%
%Do not intent the first paragraph following a heading.
%Second and subsequent paragraphs are indented by one Tab
%character (= 3 mm). If footnotes are used, they should be
%placed at the foot of the page\footnote{ Footnotes are separated
%from the text by a blank line and a printed line of length 3.5 cm.
%They should be printed in 9-point Times Roman in single line spacing.}.
%        
%\subsubsection { Third-level heading}
%Please specify references using the conventions
%illustrated below. Each should begin on a new line, and
%second and subsequent lines should be on the same page
%indented by 3 mm.

\subsection*{References}

\begin{description}

\item
Diebold, F.X., Lee, J.-H., and Weinbach, G.C. (1994). Regime switching with time-varying transition probabilities.
\textit{Business Cycles: Durations, Dynamics, and Forecasting},
144--165.

\item
Spezia, L. (2006). Bayesian analysis of non-homogeneous hidden markov models.
\textit{Journal of Statistical Computation and Simulation},
\textbf{76}(8), 713--725.

\end{description}

\end{document}





