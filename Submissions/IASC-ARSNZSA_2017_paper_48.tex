\documentclass[12pt]{article}
% \documentstyle{iascars2017}

% \usepackage{iascars2017}

\pagestyle{myheadings} 
\pagenumbering{arabic}
\topmargin 0pt \headheight 23pt \headsep 24.66pt
%\topmargin 11pt \headheight 12pt \headsep 13.66pt
\parindent = 3mm 


\begin{document}


\begin{flushleft}


{\LARGE\bf Selecting Generalised Linear Models Under Inequality Constraints}


\vspace{1.0cm}

Daniel Gerhard

\begin{description}

\item School of Mathematics \& Statistics, University of Canterbury, Christchurch, NZ

\end{description}

\end{flushleft}

%  ***** ADD ENOUGH VERTICAL SPACE HERE TO ENSURE THAT THE *****
%  ***** ABSTRACT (OR MAIN TEXT) STARTS 5 CM BELOW THE TOP *****

\vspace{0.75cm}

\noindent {\bf Abstract}. Model selection by information criteria can be used to identify a single best model or for inference based on weighted support from a set of competing models, incorporating model selection uncertainty into parameter estimates and estimates of precision. Anraku (1999) proposed a modified version of the well known Akaike information criterion, selecting models in the one-way analysis of variance models when the population means are subject to monotone trends. A generalization of this order-restricted information criterion was proposed by Kuiper et al. (2011), allowing a restriction of population means by a mixture of linear equality and inequality constraints.

An extension to this approach is presented, applying the generalized order-restricted information criterion to model selection from a set of generalized linear models. The class of models can comprise linear equality or inequality constraints of population parameters assuming a distribution of the exponential family for the response. The methodology is illustrated using the open source environment R with the add-on package \texttt{goric}.


\vskip 2mm

\noindent {\bf Keywords}.
Model selection, Order-restriction, GLM


%\section{ First-level heading}
%The C98 head 1 style leaves a half-line spacing below a
%first-level heading. There should be one blank line above
%a first-level heading.
%        
%\subsection { Second-level heading}
%There should also be one blank line above a second- or
%third-level heading (but no extra space below them).
%
%Do not intent the first paragraph following a heading.
%Second and subsequent paragraphs are indented by one Tab
%character (= 3 mm). If footnotes are used, they should be
%placed at the foot of the page\footnote{ Footnotes are separated
%from the text by a blank line and a printed line of length 3.5 cm.
%They should be printed in 9-point Times Roman in single line spacing.}.
%        
%\subsubsection { Third-level heading}
%Please specify references using the conventions
%illustrated below. Each should begin on a new line, and
%second and subsequent lines should be on the same page
%indented by 3 mm.

\subsection*{References}

\begin{description}

\item
Anraku, K. (1999).
An information criterion for parameters under a simple order restriction.
\textit{Biometrika}, \textbf{86}, 141--152.

\item
Kuiper, R.M., Hoijtink, H., Silvapulle, M.J. (2011).
An Akaike-type information criterion for model selection under inequality constraints. 
\textit{Biometrika}, \textbf{98}, 495--501.

\end{description}

\end{document}





