\documentclass[12pt]{article}
% \documentstyle{iascars2017}

% \usepackage{iascars2017}

\pagestyle{myheadings} 
\pagenumbering{arabic}
\topmargin 0pt \headheight 23pt \headsep 24.66pt
%\topmargin 11pt \headheight 12pt \headsep 13.66pt
\parindent = 3mm 


\begin{document}


\begin{flushleft}


 {\LARGE\bf An Incomplete-data Fisher Scoring Method with an Acceleration Method
}


\vspace{1.0cm}

Keiji Takai$^1$ 

\begin{description}

\item $^1 \;$ Faculty of Business and Commerce, Kansai University,
			3-3-35 Yamate, Suita, Osaka,
			JAPAN

\end{description}

\end{flushleft}

%  ***** ADD ENOUGH VERTICAL SPACE HERE TO ENSURE THAT THE *****
%  ***** ABSTRACT (OR MAIN TEXT) STARTS 5 CM BELOW THE TOP *****

\vspace{0.75cm}

\noindent {\bf Abstract}.

Incomplete data complicate conventional statistical analyses because the
analyses presume complete data are always available. The primary problem
is the complication of the parameter
estimation. The parameter estimation is based on the observed-data
log-likelihood function that consists of the sum of the logarithm of the
marginalized likelihood with respect to the missing values, and thus the
log-likelihood function becomes complicated to handle. The EM algorithm
was proposed to make it easy to handle the log-likelihood
function. However, the EM algorithm still has some problems that are
often criticized (McLachlan and Krishnan, 2002); namely, slow
convergence and unavailability of the standard error.
 

In my talk, I propose an incomplete-data Fisher scoring (IFS) method
with an acceleration method to overcome these problems.  The IFS method
takes a Newton-Raphson type iteration, but it produces exactly the
identical sequence or an approximate sequence to the sequence produced
by the EM algorithm. The notable feature of the IFS is that the IFS can
accelerate itself by adjusting its steplength and can produce the
standard error with the functions used only for the acceleration. The
convergence rate is faster than the EM algorithm. In the talk, I provide
the convergence theorem and practical examples.


\vskip 2mm

\noindent {\bf Keywords}. Incomplete data, EM algorithm, Fisher scoring, acceleration method



%\section{ First-level heading}
%The C98 head 1 style leaves a half-line spacing below a
%first-level heading. There should be one blank line above
%a first-level heading.
%        
%\subsection { Second-level heading}
%There should also be one blank line above a second- or
%third-level heading (but no extra space below them).
%
%Do not intent the first paragraph following a heading.
%Second and subsequent paragraphs are indented by one Tab
%character (= 3 mm). If footnotes are used, they should be
%placed at the foot of the page\footnote{ Footnotes are separated
%from the text by a blank line and a printed line of length 3.5 cm.
%They should be printed in 9-point Times Roman in single line spacing.}.
%        
%\subsubsection { Third-level heading}
%Please specify references using the conventions
%illustrated below. Each should begin on a new line, and
%second and subsequent lines should be on the same page
%indented by 3 mm.

\subsection*{References}


\begin{description}

\iffalse
\item
Barnett, J.A., Payne, R.W. and Yarrow, D. (1990).
\textit{Yeasts: Characteristics and identification: Second Edition.}
Cambridge: Cambridge University Press.
\fi

\item[] McLachlan, G., and Krishnan, T. (2002). The EM algorithm and
	   extensions, 2nd Edition. Wiley.

		   
		   \iffalse
\item
(ed.) Barnett, V., Payne, R. and Steiner, R. (1995).
\textit{Agricultural Sustainability: Economic, Environmental and
Statistical Considerations}. Chichester: Wiley.

\item
Payne, R.W. (1997).
\textit{Algorithm AS314 Inversion of matrices Statistics},
\textbf{46}, 295--298.

\item
Payne, R.W. and Welham, S.J. (1990).
A comparison of algorithms for combination of information in generally
balanced designs.
In: \textit{COMPSTAT90 Proceedings in Computational Statistics}, 297--302.
Heidelberg: Physica-Verlag.
\fi
\end{description}

\end{document}





