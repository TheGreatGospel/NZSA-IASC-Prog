\documentclass[12pt]{article}
% \documentstyle{iascars2017}
% \usepackage{iascars2017}

\pagestyle{myheadings} 
\pagenumbering{arabic}
\topmargin 0pt \headheight 23pt \headsep 24.66pt
%\topmargin 11pt \headheight 12pt \headsep 13.66pt
\parindent = 3mm 


\usepackage{Sweave}
\begin{document}
\input{NZSA_Abstract2-concordance}


\begin{flushleft}


{\LARGE\bf Bayesian survival analysis of batsmen in Test cricket}


\vspace{1.0cm}

O. G. Stevenson$^1$ and B. J. Brewer$^1$

\begin{description}

\item $^1 \;$ Department of Statistics, University of Auckland,
Auckland, New Zealand

\end{description}

\end{flushleft}

%  ***** ADD ENOUGH VERTICAL SPACE HERE TO ENSURE THAT THE *****
%  ***** ABSTRACT (OR MAIN TEXT) STARTS 5 CM BELOW THE TOP *****

\vspace{0.75cm}

\noindent {\bf Abstract}
It is widely accepted that in the sport of cricket, batting is more difficult
early in a player's innings, but becomes easier as a player familiarizes
themselves with the local conditions.
Here we develop a Bayesian survival analysis method to predict and quantify the
Test Match batting abilities for international cricketers, at any stage of
a player's innings.
The model is applied in two stages, firstly to individual players, allowing us
to quantify players' initial and equilibrium batting abilities, and the rate
of transition between the two.
The results indicate that most players begin a Test match innings batting with
between a quarter and a half of their potential batting ability.
The model is then implemented using a hierarchical structure, providing us with more
general inference concerning a selected group of opening batsmen from New Zealand.
Using this hierarchical structure we are able to make predictions for the batting
abilities of the next opening batsman to debut for New Zealand.
These results are considered in conjunction with other performance based
metrics, allowing us to identify players who excel in the role of opening the
batting, which has practical implications in terms of batting order and team selection policy.


\vskip 2mm

\noindent {\bf Keywords}.
Bayesian survival analysis, hierarchical modelling, cricket


%\section{ First-level heading}
%The C98 head 1 style leaves a half-line spacing below a
%first-level heading. There should be one blank line above
%a first-level heading.
%        
%\subsection { Second-level heading}
%There should also be one blank line above a second- or
%third-level heading (but no extra space below them).
%
%Do not intent the first paragraph following a heading.
%Second and subsequent paragraphs are indented by one Tab
%character (= 3 mm). If footnotes are used, they should be
%placed at the foot of the page\footnote{ Footnotes are separated
%from the text by a blank line and a printed line of length 3.5 cm.
%They should be printed in 9-point Times Roman in single line spacing.}.
%        
%\subsubsection { Third-level heading}
%Please specify references using the conventions
%illustrated below. Each should begin on a new line, and
%second and subsequent lines should be on the same page
%indented by 3 mm.

\subsection*{References}

\begin{description}

\item
Stevenson, O.G. and Brewer, B.J. (2017).
Bayesian survival anaylsis of opening batsmen in Test cricket
\textit{Journal of Quantitative Analysis in Sports}, \textit{13}(1), 25-36.

\end{description}

\end{document}





