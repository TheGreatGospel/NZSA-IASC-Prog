% !TeX spellcheck = en_NZ
\documentclass[12pt]{article}
% \documentstyle{iascars2017}

% \usepackage{iascars2017}

\pagestyle{myheadings} 
\pagenumbering{arabic}
\topmargin 0pt \headheight 23pt \headsep 24.66pt
%\topmargin 11pt \headheight 12pt \headsep 13.66pt
\parindent = 3mm 


\begin{document}


\begin{flushleft}


{\LARGE\bf  Including covariate estimation error when predicting species distributions: a simulation exercise using Template Model Builder}


\vspace{1.0cm}

Andrea Havron$^1$ and Russell Millar$^1$

\begin{description}

\item $^1 \;$ Department of Statistics, University of Auckland,
Auckland, 1010, NZ


\end{description}

\end{flushleft}

%  ***** ADD ENOUGH VERTICAL SPACE HERE TO ENSURE THAT THE *****
%  ***** ABSTRACT (OR MAIN TEXT) STARTS 5 CM BELOW THE TOP *****

\vspace{0.75cm}

\noindent {\bf Abstract}. Ecological managers often require knowledge about species distributions across a spatial region in order to facilitate best management practices. Statistical models are frequently used to infer relationships between species observations (eg. presence, abundance, biomass, etc.) and environmental covariates in order to predict values at unobserved locations. Issues remain for situations where covariate information is not available for a predictive location. In these cases, spatial maps of covariates are often generated using tools such as kriging; however, the uncertainties from this statistical estimation are not carried through to the final species distribution map. New advances in spatial modelling using the automated differentiation software, Template Model Builder, allow both the spatial process of the environmental covariates and the observations to be modelled simultaneously by maximizing the marginal likelihood of the fixed effects with a Laplace approximation after integrating out the random spatial effects. This method allows for the uncertainty of the covariate estimation process to be included in the standard errors of final predictions as well as any derived quantities, such as total biomass for a spatial region. We intend to demonstrate this method and compare our predictions to those from a model where regional covariate information is supplied from a kriging model.
\vskip 2mm

\noindent {\bf Keywords}.
spatial model, predicting covariates, Template Model Builder


%\section{ First-level heading}
%The C98 head 1 style leaves a half-line spacing below a
%first-level heading. There should be one blank line above
%a first-level heading.
%        
%\subsection { Second-level heading}
%There should also be one blank line above a second- or
%third-level heading (but no extra space below them).
%
%Do not intent the first paragraph following a heading.
%Second and subsequent paragraphs are indented by one Tab
%character (= 3 mm). If footnotes are used, they should be
%placed at the foot of the page\footnote{ Footnotes are separated
%from the text by a blank line and a printed line of length 3.5 cm.
%They should be printed in 9-point Times Roman in single line spacing.}.
%        
%\subsubsection { Third-level heading}
%Please specify references using the conventions
%illustrated below. Each should begin on a new line, and
%second and subsequent lines should be on the same page
%indented by 3 mm.

\subsection*{References}

\begin{description}

\item
Kristensen, K.,Nielsen, A., Berg, C.W., Skuag, H. and Bell, B. (2015).
TMB: Automatic Differentiation and Laplace Approximation.
In: \textit{Journal of Statistical Software},\textbf{70}, 1--21.

\end{description}

\end{document}





