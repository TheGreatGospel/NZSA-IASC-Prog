\documentclass[12pt]{article}
% \documentstyle{iascars2017}

% \usepackage{iascars2017}

\pagestyle{myheadings} 
\pagenumbering{arabic}
\topmargin 0pt \headheight 23pt \headsep 24.66pt
%\topmargin 11pt \headheight 12pt \headsep 13.66pt
\parindent = 3mm 


\begin{document}


\begin{flushleft}


{\LARGE\bf Genetic Approach and Statistical Approach for Association Study on DNA Data}


\vspace{1.0cm}

Makoto Tomita $^1$

\begin{description}

\item $^1 \;$ Clinical Research Center, Medical Hospital of Tokyo Medical and Dental University,
Tokyo, 113-8519, Japan

\end{description}

\end{flushleft}

%  ***** ADD ENOUGH VERTICAL SPACE HERE TO ENSURE THAT THE *****
%  ***** ABSTRACT (OR MAIN TEXT) STARTS 5 CM BELOW THE TOP *****

\vspace{0.75cm}

\noindent {\bf Abstract}. Genomic information such as genome-wide association analysis (GWAS) in DNA data is very large, however if the sample size corresponding to it is not enough, as an idea to solve, the author considers by a statistical approach and a genetic approach. The former will be briefly introduced, and the latter will be mainly explained. Basically, the method of focusing genome information becomes the center of presentation. 

\vskip 2mm

\noindent {\bf Keywords}.
genome wide association study, linkage disequilibrium, statistical power


%\section{ First-level heading}
%The C98 head 1 style leaves a half-line spacing below a
%first-level heading. There should be one blank line above
%a first-level heading.
%        
%\subsection { Second-level heading}
%There should also be one blank line above a second- or
%third-level heading (but no extra space below them).
%
%Do not intent the first paragraph following a heading.
%Second and subsequent paragraphs are indented by one Tab
%character (= 3 mm). If footnotes are used, they should be
%placed at the foot of the page\footnote{ Footnotes are separated
%from the text by a blank line and a printed line of length 3.5 cm.
%They should be printed in 9-point Times Roman in single line spacing.}.
%        
%\subsubsection { Third-level heading}
%Please specify references using the conventions
%illustrated below. Each should begin on a new line, and
%second and subsequent lines should be on the same page
%indented by 3 mm.

\subsection*{References}

\begin{description}

\item
Tomita, M. (2013).
Focusing Approach Using LD Block and Association Study with Haplotype Combination on DNA Data,
In: \textit{Proceedings 2013 Eleventh International Conference on ICT and Knowledge Engineering}, 5--10.
Bangkok: IEEE Conference \#32165.

\item
Tomita, M. (2015).
Haplotype estimation, haplotype block identification and statistical analysis for DNA data,
In: \textit{Conference Program and Book of Abstracts, Conference of the International Federation of Classication Societies (IFCS-2015)}, 227--228, Bologna.

\item
Tomita, M., Hatsumichi, M. and Kurihara, K. (2008).
\textit{Computational Statistics and Data Analysis},
\textbf{52}(4), 1806--1820.

\item
Tomita, M., Hashimoto, N. and Tanaka, Y. (2011).
\textit{Computational Statistics and Data Analysis},
\textbf{55}(6), 2104--2113.

\item
Tomita, M., Kubota, T. and Ishioka, F. (2015).
\textit{PLoS ONE},
\textbf{10}(7), e0127358.

\end{description}

\end{document}





