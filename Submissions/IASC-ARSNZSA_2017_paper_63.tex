\documentclass[12pt]{article}
% \documentstyle{iascars2017}

% \usepackage{iascars2017}

\pagestyle{myheadings} 
\pagenumbering{arabic}
\topmargin 0pt \headheight 23pt \headsep 24.66pt
%\topmargin 11pt \headheight 12pt \headsep 13.66pt
\parindent = 3mm 


\begin{document}


\begin{flushleft}


{\LARGE\bf A Max-Type Multivariate Two-Sample Baumgartner Statistic}


\vspace{1.0cm}

Hidetoshi Murakami$^1$
\begin{description}

\item $^1 \;$ Department of Applied Mathematics, Tokyo University of Science \\ 1-3 Kagurazaka, Shinjyuku-ku, Tokyo, 162-8601, Japan

\end{description}

\end{flushleft}

%  ***** ADD ENOUGH VERTICAL SPACE HERE TO ENSURE THAT THE *****
%  ***** ABSTRACT (OR MAIN TEXT) STARTS 5 CM BELOW THE TOP *****

\vspace{0.75cm}

\noindent {\bf Abstract}. 
A multivariate two-sample testing problem is one of the most important topics in nonparametric statistics. Further, a max-type Baumgartner statistic based on the modified Baumgartner statistic (Murakami, 2006) was proposed by Murakami (2012) for testing the equality of two continuous distribution functions. In this paper, a max-type multivariate two-sample Baumgartner statistic is suggested based on the Jure\v{c}kov\'{a} and Kalina's ranks of distances (Jure\v{c}kov\'{a} and Kalina, 2012). Simulations are used to investigate the power of the suggested statistic for various population distributions. The results indicate that the proposed test statistic is more suitable than various existing statistics for testing a shift in the location, scale and location-scale parameters. 

\vskip 2mm

\noindent {\bf Keywords}.
Baumgartner statistic, Jure\v{c}kov\'{a} \& Kalina's ranks of distances, Multivariate two-sample rank test, Power comparison


%\section{ First-level heading}
%The C98 head 1 style leaves a half-line spacing below a
%first-level heading. There should be one blank line above
%a first-level heading.
%        
%\subsection { Second-level heading}
%There should also be one blank line above a second- or
%third-level heading (but no extra space below them).
%
%Do not intent the first paragraph following a heading.
%Second and subsequent paragraphs are indented by one Tab
%character (= 3 mm). If footnotes are used, they should be
%placed at the foot of the page\footnote{ Footnotes are separated
%from the text by a blank line and a printed line of length 3.5 cm.
%They should be printed in 9-point Times Roman in single line spacing.}.
%        
%\subsubsection { Third-level heading}
%Please specify references using the conventions
%illustrated below. Each should begin on a new line, and
%second and subsequent lines should be on the same page
%indented by 3 mm.

\subsection*{References}

\begin{description}
\item Jure\v{c}kov\'{a}, J. and Kalina, J. (2012). 
Nonparametric multivariate rank tests and their unbiasedness. 
{\textit{Bernoulli}}, \textbf{18}, 229--251.

\item Murakami, H. (2006). 
A $k$-sample rank test based on the modified Baumgartner statistic and its power comparison. 
{\textit{Journal of the Japanese Society of Computational Statistics}}, {\textbf{19}}, 1--13.

\item Murakami, H. (2012). 
A max-type Baumgartner statistic for the two-sample problem and its power comparison. 
{\textit{Journal of the Japanese Society of Computational Statistics}}, \textbf{25}, 39--49.



\end{description}

\end{document}





