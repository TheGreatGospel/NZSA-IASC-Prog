\documentclass[10.9pt]{article}
% \documentstyle{iascars2017}

% \usepackage{iascars2017}

\pagestyle{myheadings} 
\pagenumbering{arabic}
\topmargin 0pt \headheight 23pt \headsep 24.66pt
%\topmargin 11pt \headheight 12pt \headsep 13.66pt
\parindent = 3mm 


\begin{document}


\begin{flushleft}


{\LARGE\bf Bayesian Static Parameter Inference for Partially Observed Stochastic Systems}


\vspace{1.0cm}

YAXIAN XU$^1$ and AJAY JASRA$^1$

\begin{description}

\item $^1 \;$ Department of Statistics \& Applied Probability, National University of Singapore, Singapore, 117546, SG.


\end{description}

\end{flushleft}

%  ***** ADD ENOUGH VERTICAL SPACE HERE TO ENSURE THAT THE *****
%  ***** ABSTRACT (OR MAIN TEXT) STARTS 5 CM BELOW THE TOP *****

\vspace{0.75cm}

\noindent {\bf Abstract}. We consider Bayesian static parameter estimation for partially observed stochastic systems with discrete-time observations. This is a very important problem, but is very computationally challenging as the associated posterior distributions are highly complex and one has to resort to discretizing the associated probability law of the underlying stochastic system and advanced Markov chain Monte Carlo (MCMC) techniques to infer the parameters. We are interested in the situation where the discretization is in multiple dimensions. For instance, for partially observed stochastic partial differential equations (SPDEs), where dicretization is in both space and time. In such cases, multi-index Monte Carlo (MIMC) is known to have the potential to reduce the computational cost for a prescribed level of error, relative to i.i.d. sampling from the most precise discretization. We demonstrate how MCMC and particularly particle MCMC can be used in the multi-index framework for Bayesian static parameter inference for the above-mentioned models. The main idea involves constructing an approximate coupling of the posterior density of the joint on the parameter and hidden space and then correcting by an importance sampling method. Our method is illustrated numerically to be preferable for inference of parameters for a partially observed SPDE.

\vskip 2mm

\noindent {\bf Keywords}.
Multi-index Monte Carlo, Markov chain Monte Carlo, stochastic partial differential equations


%\section{ First-level heading}
%The C98 head 1 style leaves a half-line spacing below a
%first-level heading. There should be one blank line above
%a first-level heading.
%        
%\subsection { Second-level heading}
%There should also be one blank line above a second- or
%third-level heading (but no extra space below them).
%
%Do not intent the first paragraph following a heading.
%Second and subsequent paragraphs are indented by one Tab
%character (= 3 mm). If footnotes are used, they should be
%placed at the foot of the page\footnote{ Footnotes are separated
%from the text by a blank line and a printed line of length 3.5 cm.
%They should be printed in 9-point Times Roman in single line spacing.}.
%        
%\subsubsection { Third-level heading}
%Please specify references using the conventions
%illustrated below. Each should begin on a new line, and
%second and subsequent lines should be on the same page
%indented by 3 mm.

\subsection*{References}

\begin{description}
\item
Christophe Andrieu, Arnaud Doucet, and Roman Holenstein. (2010).
Particle markov chain monte carlo methods.
\textit{Journal of the Royal Statistical Society: Series B (Statistical Methodology)}, 72(3):269--342.

\item
Haji-Ali, A. L., Nobile, F. \& Tempone, R. (2016).
Multi-Index Monte Carlo: When sparsity meets sampling.
\textit{Numerische Mathematik}, 132, 767--806.
\end{description}

\end{document}





